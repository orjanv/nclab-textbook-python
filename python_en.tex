\documentclass[article,A4,12pt]{llncs}

% Use of Pygments
% pygmentize -f latex -O full test.py > p

% Conditional compilation.
% NOTE: If you set fullversionfalse, just compile ONCE so that TOC stays unchanged.
\newif\iffullversion
\fullversiontrue
%\fullversionfalse

\usepackage[T1]{fontenc}
\usepackage{amsmath}
\usepackage{amssymb}
\usepackage{color}
\usepackage{amsfonts}
\usepackage{mathrsfs, bm}

\usepackage{graphicx}
\usepackage{tabularx}
\usepackage{subfig}
\usepackage{epsf,times}
\usepackage{color}
\usepackage{wrapfig}
\usepackage{cases}
\usepackage{multicol}
\usepackage[usenames,dvipsnames]{xcolor}

\usepackage{palatino}

\usepackage[T1]{fontenc}
%\newcommand{\tmname}[1]{\textsc{#1}}
%\newcommand{\tmop}[1]{\ensuremath{\operatorname{#1}}}
%\newcommand{\tmsamp}[1]{\textsf{#1}}
%\newcommand{\tmtextsc}[1]{{\scshape{#1}}}
%\newcommand{\tmtextsl}[1]{{\slshape{#1}}}
%\newcommand{\tmtexttt}[1]{{\ttfamily{#1}}}

\leftmargin=0.0cm
\oddsidemargin=0.5cm
\evensidemargin=0.5cm
\topmargin=0cm
\textwidth=16.0cm
%\textheight=21.5cm
\textheight=20.0cm
\pagestyle{plain}
\setlength{\columnsep}{20pt}

\def\m{\mathbf{m}}
\def\H{\mathbf{H}}
\def\E{\mathbf{E}}
\newcommand{\vepsi}{{\varepsilon}}
\def\hnorm#1#2{\vert\,#1\,\vert_{#2}}
\newcommand{\R}{{\mathbb R}}
\newcommand{\Sph}{{\mathbb S}}
\def\x{\mathbf{x}}
\def\hvec{\overline{\mathbf{h}}}
\def\evec{\overline{\mathbf{e}}}

\newcommand{ \etal}{\mbox{\emph{et al. }}}

\newcommand\vect[1]{\mbf{#1}}
\newcommand{\mbf}[1]{\mbox{\boldmath$#1$}} 
\newcommand{\RC}[1]{#1 $\times$ #1 $\times$ #1}
\def\um{$\mu$m}
\def\C{$^{\circ}\mathrm{C}$}

\newcommand{\Rmnum}[1]{\expandafter\@slowromancap\romannumeral #1@}

% DEFINITION OF CUSTOM FONT SIZE
\newcommand{\customfontA}{\fontsize{50}{55}\selectfont}
\newcommand{\customfontB}{\fontsize{14.4}{20}\selectfont}
\newcommand{\customfontC}{\fontsize{30}{35}\selectfont}

\DeclareMathAlphabet{\mathpzc}{OT1}{pzc}{m}{it}

\def\clovek#1{\noindent\bgroup\vbox{\noindent#1}\egroup\vskip1em}

% TO INPUT BACKGROUND IMAGE
%\usepackage{eso-pic}
%\newcommand\BackgroundPic{
%\put(0,0){
%\parbox[b][\paperheight]{\paperwidth}{
%\vfill
%\centering
%\includegraphics[width=\paperwidth,height=\paperheight]{img/karel-frontpage.png}
%%\includegraphics[width=\paperwidth,height=\paperheight]{img/background.jpg}
%\vfill
%}}}

\usepackage{fancyvrb}

\newenvironment{bluecode}{\VerbatimEnvironment \color{blue} \begin{Verbatim}}
{\end{Verbatim}}
\newenvironment{greencode}{\VerbatimEnvironment \color{ForestGreen} \begin{Verbatim}}
{\end{Verbatim}}
\newenvironment{redcode}{\VerbatimEnvironment \color{Red} \begin{Verbatim}}
{\end{Verbatim}}

% For Pygments:
\usepackage{fancyvrb}
\usepackage{color}
\usepackage[utf-8]{inputenc}

\makeatletter
\def\PY@reset{\let\PY@it=\relax \let\PY@bf=\relax%
    \let\PY@ul=\relax \let\PY@tc=\relax%
    \let\PY@bc=\relax \let\PY@ff=\relax}
\def\PY@tok#1{\csname PY@tok@#1\endcsname}
\def\PY@toks#1+{\ifx\relax#1\empty\else%
    \PY@tok{#1}\expandafter\PY@toks\fi}
\def\PY@do#1{\PY@bc{\PY@tc{\PY@ul{%
    \PY@it{\PY@bf{\PY@ff{#1}}}}}}}
\def\PY#1#2{\PY@reset\PY@toks#1+\relax+\PY@do{#2}}

\def\PY@tok@gd{\def\PY@tc##1{\textcolor[rgb]{0.63,0.00,0.00}{##1}}}
\def\PY@tok@gu{\let\PY@bf=\textbf\def\PY@tc##1{\textcolor[rgb]{0.50,0.00,0.50}{##1}}}
\def\PY@tok@gt{\def\PY@tc##1{\textcolor[rgb]{0.00,0.25,0.82}{##1}}}
\def\PY@tok@gs{\let\PY@bf=\textbf}
\def\PY@tok@gr{\def\PY@tc##1{\textcolor[rgb]{1.00,0.00,0.00}{##1}}}
\def\PY@tok@cm{\let\PY@it=\textit\def\PY@tc##1{\textcolor[rgb]{0.25,0.50,0.50}{##1}}}
\def\PY@tok@vg{\def\PY@tc##1{\textcolor[rgb]{0.10,0.09,0.49}{##1}}}
\def\PY@tok@m{\def\PY@tc##1{\textcolor[rgb]{0.40,0.40,0.40}{##1}}}
\def\PY@tok@mh{\def\PY@tc##1{\textcolor[rgb]{0.40,0.40,0.40}{##1}}}
\def\PY@tok@go{\def\PY@tc##1{\textcolor[rgb]{0.50,0.50,0.50}{##1}}}
\def\PY@tok@ge{\let\PY@it=\textit}
\def\PY@tok@vc{\def\PY@tc##1{\textcolor[rgb]{0.10,0.09,0.49}{##1}}}
\def\PY@tok@il{\def\PY@tc##1{\textcolor[rgb]{0.40,0.40,0.40}{##1}}}
\def\PY@tok@cs{\let\PY@it=\textit\def\PY@tc##1{\textcolor[rgb]{0.25,0.50,0.50}{##1}}}
\def\PY@tok@cp{\def\PY@tc##1{\textcolor[rgb]{0.74,0.48,0.00}{##1}}}
\def\PY@tok@gi{\def\PY@tc##1{\textcolor[rgb]{0.00,0.63,0.00}{##1}}}
\def\PY@tok@gh{\let\PY@bf=\textbf\def\PY@tc##1{\textcolor[rgb]{0.00,0.00,0.50}{##1}}}
\def\PY@tok@ni{\let\PY@bf=\textbf\def\PY@tc##1{\textcolor[rgb]{0.60,0.60,0.60}{##1}}}
\def\PY@tok@nl{\def\PY@tc##1{\textcolor[rgb]{0.63,0.63,0.00}{##1}}}
\def\PY@tok@nn{\let\PY@bf=\textbf\def\PY@tc##1{\textcolor[rgb]{0.00,0.00,1.00}{##1}}}
\def\PY@tok@no{\def\PY@tc##1{\textcolor[rgb]{0.53,0.00,0.00}{##1}}}
\def\PY@tok@na{\def\PY@tc##1{\textcolor[rgb]{0.49,0.56,0.16}{##1}}}
\def\PY@tok@nb{\def\PY@tc##1{\textcolor[rgb]{0.00,0.50,0.00}{##1}}}
\def\PY@tok@nc{\let\PY@bf=\textbf\def\PY@tc##1{\textcolor[rgb]{0.00,0.00,1.00}{##1}}}
\def\PY@tok@nd{\def\PY@tc##1{\textcolor[rgb]{0.67,0.13,1.00}{##1}}}
\def\PY@tok@ne{\let\PY@bf=\textbf\def\PY@tc##1{\textcolor[rgb]{0.82,0.25,0.23}{##1}}}
\def\PY@tok@nf{\def\PY@tc##1{\textcolor[rgb]{0.00,0.00,1.00}{##1}}}
\def\PY@tok@si{\let\PY@bf=\textbf\def\PY@tc##1{\textcolor[rgb]{0.73,0.40,0.53}{##1}}}
\def\PY@tok@s2{\def\PY@tc##1{\textcolor[rgb]{0.73,0.13,0.13}{##1}}}
\def\PY@tok@vi{\def\PY@tc##1{\textcolor[rgb]{0.10,0.09,0.49}{##1}}}
\def\PY@tok@nt{\let\PY@bf=\textbf\def\PY@tc##1{\textcolor[rgb]{0.00,0.50,0.00}{##1}}}
\def\PY@tok@nv{\def\PY@tc##1{\textcolor[rgb]{0.10,0.09,0.49}{##1}}}
\def\PY@tok@s1{\def\PY@tc##1{\textcolor[rgb]{0.73,0.13,0.13}{##1}}}
\def\PY@tok@sh{\def\PY@tc##1{\textcolor[rgb]{0.73,0.13,0.13}{##1}}}
\def\PY@tok@sc{\def\PY@tc##1{\textcolor[rgb]{0.73,0.13,0.13}{##1}}}
\def\PY@tok@sx{\def\PY@tc##1{\textcolor[rgb]{0.00,0.50,0.00}{##1}}}
\def\PY@tok@bp{\def\PY@tc##1{\textcolor[rgb]{0.00,0.50,0.00}{##1}}}
\def\PY@tok@c1{\let\PY@it=\textit\def\PY@tc##1{\textcolor[rgb]{0.25,0.50,0.50}{##1}}}
\def\PY@tok@kc{\let\PY@bf=\textbf\def\PY@tc##1{\textcolor[rgb]{0.00,0.50,0.00}{##1}}}
\def\PY@tok@c{\let\PY@it=\textit\def\PY@tc##1{\textcolor[rgb]{0.25,0.50,0.50}{##1}}}
\def\PY@tok@mf{\def\PY@tc##1{\textcolor[rgb]{0.40,0.40,0.40}{##1}}}
\def\PY@tok@err{\def\PY@bc##1{\fcolorbox[rgb]{1.00,0.00,0.00}{1,1,1}{##1}}}
\def\PY@tok@kd{\let\PY@bf=\textbf\def\PY@tc##1{\textcolor[rgb]{0.00,0.50,0.00}{##1}}}
\def\PY@tok@ss{\def\PY@tc##1{\textcolor[rgb]{0.10,0.09,0.49}{##1}}}
\def\PY@tok@sr{\def\PY@tc##1{\textcolor[rgb]{0.73,0.40,0.53}{##1}}}
\def\PY@tok@mo{\def\PY@tc##1{\textcolor[rgb]{0.40,0.40,0.40}{##1}}}
\def\PY@tok@kn{\let\PY@bf=\textbf\def\PY@tc##1{\textcolor[rgb]{0.00,0.50,0.00}{##1}}}
\def\PY@tok@mi{\def\PY@tc##1{\textcolor[rgb]{0.40,0.40,0.40}{##1}}}
\def\PY@tok@gp{\let\PY@bf=\textbf\def\PY@tc##1{\textcolor[rgb]{0.00,0.00,0.50}{##1}}}
\def\PY@tok@o{\def\PY@tc##1{\textcolor[rgb]{0.40,0.40,0.40}{##1}}}
\def\PY@tok@kr{\let\PY@bf=\textbf\def\PY@tc##1{\textcolor[rgb]{0.00,0.50,0.00}{##1}}}
\def\PY@tok@s{\def\PY@tc##1{\textcolor[rgb]{0.73,0.13,0.13}{##1}}}
\def\PY@tok@kp{\def\PY@tc##1{\textcolor[rgb]{0.00,0.50,0.00}{##1}}}
\def\PY@tok@w{\def\PY@tc##1{\textcolor[rgb]{0.73,0.73,0.73}{##1}}}
\def\PY@tok@kt{\def\PY@tc##1{\textcolor[rgb]{0.69,0.00,0.25}{##1}}}
\def\PY@tok@ow{\let\PY@bf=\textbf\def\PY@tc##1{\textcolor[rgb]{0.67,0.13,1.00}{##1}}}
\def\PY@tok@sb{\def\PY@tc##1{\textcolor[rgb]{0.73,0.13,0.13}{##1}}}
\def\PY@tok@k{\let\PY@bf=\textbf\def\PY@tc##1{\textcolor[rgb]{0.00,0.50,0.00}{##1}}}
\def\PY@tok@se{\let\PY@bf=\textbf\def\PY@tc##1{\textcolor[rgb]{0.73,0.40,0.13}{##1}}}
\def\PY@tok@sd{\let\PY@it=\textit\def\PY@tc##1{\textcolor[rgb]{0.73,0.13,0.13}{##1}}}

\def\PYZbs{\char`\\}
\def\PYZus{\char`\_}
\def\PYZob{\char`\{}
\def\PYZcb{\char`\}}
\def\PYZca{\char`\^}
\def\PYZsh{\char`\#}
\def\PYZpc{\char`\%}
\def\PYZdl{\char`\$}
\def\PYZti{\char`\~}
% for compatibility with earlier versions
\def\PYZat{@}
\def\PYZlb{[}
\def\PYZrb{]}
\makeatother
% End of Pygments inputs.

% Define color boxes:
\definecolor{MyGreen}{rgb}{0.9, 1, 0.9}
\makeatletter\newenvironment{gbox}{%
   \begin{lrbox}{\@tempboxa}\begin{minipage}{0.985\columnwidth}}{\end{minipage}\end{lrbox}%
   \noindent
   \colorbox{MyGreen}{\usebox{\@tempboxa}}
}\makeatother

%\definecolor{MyYellow}{rgb}{0.98, 0.98, 0.824}
\definecolor{MyYellow}{rgb}{1, 0.99, 0.8}
\makeatletter\newenvironment{ybox}{%
   \begin{lrbox}{\@tempboxa}\begin{minipage}{0.985\columnwidth}}
   {\end{minipage}\end{lrbox}%
   \noindent
   \colorbox{MyYellow}{\usebox{\@tempboxa}}
}\makeatother

\definecolor{MyBlue}{rgb}{0.88, 0.95, 1}
\makeatletter\newenvironment{bbox}{%
   \begin{lrbox}{\@tempboxa}\begin{minipage}{0.985\columnwidth}}
   {\end{minipage}\end{lrbox}%
   \noindent
   \colorbox{MyBlue}{\usebox{\@tempboxa}}
}\makeatother

\usepackage{wallpaper}

\begin{document}

\ThisULCornerWallPaper{1.02}{img/python-cover.png}

%%%%%%%%%%%%%%%%%%%%%%%%%%%%%%%%%%%%%%%%%%%%%%%%%%%%%%%%%%%%%%%%%%%%%%%%%
\pagestyle{empty}
\newpage
\vbox{}
\newpage
\vbox{}
\vfill

\centerline{Revision March-25-2013}

%%%%%%%%%%%%%%%%%%%%%%%%%%%%%%%%%%%%%%%%%%%%%%%%%%%%%%%%%%%%%%%%%%%%%%%%%
\newpage

\noindent
{\bf About this Textbook}\\[4mm]
This free open source textbook is provided as a courtesy to NCLab users. 
Python is a modern high-level 
dynamic programming language that is used in many areas of business, 
engineering, and science today. After taking this course, you will 
have solid theoretical knowledge and vast practical experience with 
computer programming in Python. \\[4mm]

\noindent
{\bf Become a Co-Author}\\[4mm]
We do not plan to publish the textbook with a commercial publisher since this 
would make it unnecessarily expensive for kids and students who are the main 
target audience. Feel free to contribute to the textbook with any material or 
suggestions. There is never enough illustrations and exercises, and there always 
are bugs to report. Translating the textbook into other languages would benefit
thousands of kids worldwide. Instructions for contributors can be found 
below.\\[4mm]

\noindent
{\bf How to Contribute (for \LaTeX \ and Git users)}\\[4mm]
\noindent
The textbook is written in \LaTeX, a high-quality typesetting system that 
you can learn and use in NCLab. In the future it will be possible to contribute to 
the textbook directly in NCLab, but at this time, the sources are stored 
in a public Git repository {\tt nclab-textbook-python} at Github (http://github.com). \\[4mm]

\noindent
{\bf How to Contribute (for all others)}\\[4mm]
\noindent
We will gladly accept new interesting exercises as well as
new images of good quality that will make the textbook more interesting and fun. You
can send those at any time via email to {\tt pavel@femhub.com}.\\[4mm]

\noindent
{\bf List of Co-Authors}
\begin{itemize}
\item Pavel Solin, University of Nevada, Reno (primary author), USA. 
\item Martin Novak, Czech Technical University, Prague, Czech Republic.
\item Salih Dede, Coral Academy of Science High School, Reno, USA.
\item Nazhmiddin Shapoatov, Sonoran Science Academy, Phoenix, USA.
\item Joel Landsteiner, Cray Inc, USA.
\item William Mitchell, NIST, USA.
\item Venkata Rama Rao Mallela, Hyderabad, India.
\item Steven Lamb, Philadelphia, USA.
\item Norman Dunbar, Leeds, West Yorkshire, England.
\item Samuel Marks, Sydney, Australia.
\end{itemize}
\vspace{4mm}
{\bf Graphics Design:} TR-Design {\tt http://tr-design.cz}

%\vspace{6mm}
%\noindent
%{\bf For Instructors}\\[4mm]
%Review Book and Exercise Book containing 
%review questions with answers and programming exercises with
%solutions, are part of the NCLab-powered course 
%{\em Intro to Programming with Karel the Robot and Python} that is 
%available at \\[4mm]
%
%{\color{blue}
%\centerline{\tt http://introtoprogramming.net}
%}
%\vspace{5mm}
%
%\noindent
%for a small subscription fee. The fee is used to cover cloud computing resources, 
%development, maintenance, and user support. 
%
%The course is completely web-browser based, no installation of anything at your school 
%or home is needed. You and your students can access the course from anywhere and at any 
%time. Instructor's workflow includes downloading assignments and review
%question worksheets from the database, sending them to the students 
%via one mouse click, and collecting them back, automatically graded. 
%The course is scheduled to open in January 2013. In the meantime, enjoy 
%Karel and Python, and let us know with any questions at {\tt support@nclab.com}!



\newpage
%{\ }
\setcounter{tocdepth}{2}
\tableofcontents
%\pagestyle{plain}

\newpage

\pagestyle{plain}
\setcounter{page}{1}

%%%%%%%%%%%%%%%%%%%%%%%%%%%%%%%%%%%%%%%%%%%%%%%%%%%%%%%%%%%%%%%%%%%%%%%%%
\pagestyle{plain}
\setcounter{page}{1}
\section*{Foreword}
This course provides a gentle yet efficient introduction to Python -- 
a modern high-level dynamic programming language that is widely used in business, 
science, and engineering applications. If this is your first time learning 
a programming language, then we recommend that you spend a few days with 
Karel the Robot before diving into Python. Karel the Robot is available 
in NCLab (http://nclab.com) and it will teach you what you will need most 
-- {\em algorithmic thinking}. 

Algorithmic thinking is ability to translate your ideas into 
procedures, or sequences of steps, that are compatible with 
the way a machine operates. It is the most essential 
skill in computer programming. Algorithmic thinking is not 
bound to any specific programming language, and therefore it is 
a very good idea to acquire it using a very simple language 
such as Karel the Robot. Once you have algorithmic thinking,  
your efficiency in learning any new programming language
will improve dramatically. Moreover, Karel's syntax is very similar to Python, 
so the transition from Karel to Python is effortless.

In Python you will learn more applied concepts including mathematical operations
and functions, 2D and 3D plotting, local and global variables, strings, tuples, lists, dictionaries, 
exceptions, object-oriented programming, and more. 
A strong companion of Python are its libraries. Besides the Standard Library that
contains many built-in functions not present in lower-level languages such as 
Java, C, C++ or Fortran, Python also has powerful scientific libraries including 
Scipy, Numpy, Matplotlib, Pylab, Sympy and others. With these, you will be able to 
solve entry-level scientific and engineering problems. These libraries are used
throughout the course.  

%%%%%%%%%%%%%%%%%%%%%%%%%%%%%%%%%%%%%%%%%%%%%%%%%%%%%%%%%%%%%%%%%%%%%%

\part{Textbook}

\setcounter{section}{0}

\input textbook.tex

%%%%%%%%%%%%%%%%%%%%%%%%%%%%%%%%%%%%%%%%%%%%%%%%%%%%%%%%%%%%%%%%%%%%%%

\part{Programming Exercises}

\setcounter{section}{0}

\input exercises.tex

%%%%%%%%%%%%%%%%%%%%%%%%%%%%%%%%%%%%%%%%%%%%%%%%%%%%%%%%%%%%%%%%%%%%%%

\part{Review Questions}

\input review-questions.tex

\end{document}

%%%%%%%%%%%%%%%%%%%%%%%%%%%%%%%%%%%%%%%%%%%%%%%%%%%%%%%%%%%%%%
%%%                                                        %%%
%%%   IMPORTANT - DO NOT ADD REVIEW QUESTIONS HERE         %%%
%%%                                                        %%%
%%%   REVIEW QUESTIONS MUST FIRST GO TO SOLUTION MANUAL,   %%%
%%%   THEN STRIPPED OFF SOLUTIONS AND GO HERE              %%%
%%%                                                        %%%
%%%%%%%%%%%%%%%%%%%%%%%%%%%%%%%%%%%%%%%%%%%%%%%%%%%%%%%%%%%%%%

\setcounter{section}{0}
\section{Introduction}

\begin{enumerate}
\item {\em What is the difference between compiled and scripting programming languages?}\\

\begin{enumerate}
\item[A1] Programs written in compiled languages are binary files, programs written in 
          scripting languages are text files. 
\item[A2] Compiled languages are easier to use than scripting languages.
\item[A3] Programs written in scripting languages are usually more efficient than programs 
          written in compiled languages.
\item[A4] Scripting languages do not require a compiler.
\end{enumerate}
\vspace{4mm}

\item {\em What languages take better advantage of the underlying hardware architecture
      and why?}\\

\begin{enumerate}
\item[A1] Scripting languages because they are less hardware-dependent.
\item[A2] Compiled languages because the executable files are tailored 
          to the concrete hardware.
\item[A3] Compiled languages because they do not require a linker.
\item[A4] Scripting languages because they do not require a compiler.
\end{enumerate}
\vspace{4mm}

\item {\em Give three examples of compiled programming languages.}\\

\begin{enumerate}
\item[A1] Fortran, C++ and Lua.
\item[A2] Python, Ruby and C.
\item[A3] C, C++ and Fortran.
\item[A4] Python, Lua and Perl.
\end{enumerate}
\vspace{4mm}

\item {\em Give three examples of interpreted programming languages.}\\

\begin{enumerate}
\item[A1] Pascal, Perl and Python.
\item[A2] Lua, C and C++.
\item[A3] Perl, Python and Ruby. 
\item[A4] C, C++ and Fortran.
\end{enumerate}
\vspace{4mm}

\item {\em Where does the name "Python" of the programming language come from?}\\

\begin{enumerate}
\item[A1] The snake, {\em Python regius}.
\item[A2] TV show in Great Britain.
\item[A3] Brand name of remote start systems.
\item[A4] Brand name of aquarium products.
\end{enumerate}
\vspace{4mm}

\item {\em When was the implementation of Python started?}\\

\begin{enumerate}
\item[A1] 1989
\item[A2] 1995
\item[A3] 2000
\item[A4] 2005
\end{enumerate}
\vspace{4mm}

\item {\em Name three programming styles that Python permits.}\\

\begin{enumerate}
\item[A1] Compiled, interpreted, scripting.
\item[A2] Procedural, object-oriented, functional.
\item[A3] Procedural, object-oriented, binary.
\item[A4] Compact, literal, extended.
\end{enumerate}
\vspace{4mm}

\item {\em How can displayed Python projects be cloned?}\\

\begin{enumerate}
\item[A1] In Python worksheet through the {\em File} menu.
\item[A2] Through File Manager's {\em Project} menu.
\item[A3] Using the icon {\em Displayed projects} on Desktop.
\item[A4] Python projects cannot be cloned.
\end{enumerate}
\vspace{4mm}

\item {\em How can new Python project be launched?}\\

\begin{enumerate}
\item[A1] Through the {\em Program} menu.
\item[A2] Through File Manager's {\em Settings} menu.
\item[A3] Through File Manager's {\em Project} menu.
\item[A4] Through the Programming module on Desktop.
\end{enumerate}
\vspace{4mm}

\item {\em How are Python programs processed in NCLab?}\\

\begin{enumerate}
\item[A1] They are translated into Javascript and run in your web browser.
\item[A2] They are interpreted on a remote server.
\item[A3] They are interpreted on your PC computer / laptop / tablet.
\item[A4] They are translated into Flash and run in your web browser.
\end{enumerate}
\vspace{4mm}

\item {\em What are the types of cells that a Python worksheet can contain?}\\

\begin{enumerate}
\item[A1] Error message cells.
\item[A2] Output cells.
\item[A3] Code cells.
\item[A4] Descriptive cells.
\end{enumerate}
\vspace{4mm}

\item {\em How can all code cells in a Python worksheet be evaluated at once?}\\

\begin{enumerate}
\item[A1] Type Evaluate in the last code cell and hit ENTER.
\item[A2] Click on the green arrow under the last code cell.
\item[A3] Click on {\em Evaluate all} in the {\em File} menu. 
\item[A4] Click on the green arrow button in the upper menu.
\end{enumerate}
\vspace{4mm}

\item {\em If your Python worksheet has multiple code cells, 
how can a single code cell be evaluated?}\\

\begin{enumerate}
\item[A1] Click on {\tt save} under the code cell.
\item[A2] Click on the green arrow button in the upper menu.
\item[A3] Click on the green arrow under the code cell.
\item[A4] Click on the red button in the menu.
\end{enumerate}
\vspace{4mm}

\item {\em What is the way to add a new code cell?}\\

\begin{enumerate}
\item[A1] Via a button located under any code cell or descriptive cell. 
\item[A2] Click on {\em New} in the {\em File} menu.
\item[A3] Click on the larger "A" icon in the menu.
\item[A4] Click on {\em New code cell} in {\em Edit} menu.
\end{enumerate}
\vspace{4mm}

\item {\em How can a new descriptive cell be added?}\\

\begin{enumerate}
\item[A1] Click on the smaller "A" icon in the menu.
\item[A2] Click on {\em New descriptive cell} in the {\em Edit} menu.
\item[A3] Click on {\em New} in the {\em File} menu.
\item[A4] Via a button located under any code cell or descriptive cell. 
\end{enumerate}
\vspace{4mm}

\item {\em What is the correct way to remove a code, output, or descriptive cell?}\\

\begin{enumerate}
\item[A1] Cells, once created, cannot be removed.
\item[A2] Via a button located under the cell.
\item[A3] Click on {\em Remove active cell} in the {\em Edit} menu.
\item[A4] Click on {\em Close} in the {\em File} menu.
\end{enumerate}
\vspace{4mm}

\item {\em Which of the following are scientific libraries for Python?}\\

\begin{enumerate}
\item[A1] GNU Scientific Library
\item[A2] Numpy
\item[A3] Scipy
\item[A4] Sympy
\end{enumerate}
\vspace{4mm}

\item {\em What is the correct way to print the text string "I like Python" ?}\\

\begin{enumerate}
\item[A1] {\tt print "I like Python"}
\item[A2] {\tt write "I like Python"} 
\item[A3] {\tt write 'I like Python'} 
\item[A4] {\tt write >>I like Python<<} 
\end{enumerate}
\vspace{4mm}

\end{enumerate}

%%%%%%%%%%%%%%%%%%%%%%%%%%%%%%%%%%%%%%%%%%%%%%%%%%%%%%%%%%%%%%%%%%%%%

\section{Using Python as a Calculator}

\begin{enumerate}

\item {\em What will be the result of {\tt 5 * 4 - 3 * 2} ?}\\

\begin{enumerate}
\item[A1] {\tt 34}
\item[A2] {\tt 10}
\item[A3] {\tt 0}
\item[A4] {\tt 14}
\end{enumerate}

\vspace{6mm}

\item {\em What will be the result of {\tt 11/4} ?}\\

\begin{enumerate}
\item[A1] {\tt 2.75}
\item[A2] {\tt 2}
\item[A3] {\tt 3}
\item[A4] Error message.
\end{enumerate}

\vspace{6mm}

\item {\em Type $3^2$ using Python syntax!}\\

\begin{enumerate}
\item[A1] {\tt 3\^{}2}
\item[A2] {\tt 3\^{}\^{}2}
\item[A3] {\tt 3*\^{}2}
\item[A4] {\tt 3**2}
\end{enumerate}

\vspace{6mm}

\item {\em What will be the result of {\tt (-2)**3} ?}\\

\begin{enumerate}
\item[A1] {\tt 2}
\item[A2] {\tt 8}
\item[A3] {\tt -8}
\item[A4] Error message
\end{enumerate}

\vspace{6mm}

\item {\em What will be the result of {\tt (-2)**(3.5)} ?}\\

\begin{enumerate}
\item[A1] {\tt -8}
\item[A2] {\tt 11.313708498984761}
\item[A3] {\tt -11.313708498984761}
\item[A4] Error message
\end{enumerate}

\vspace{6mm}

\item {\em What does the {\em modulo} operation do ?}\\

\begin{enumerate}
\item[A1] Rounding up
\item[A2] Remainder after integer division
\item[A3] Integer division
\item[A4] Rounding down
\end{enumerate}

\vspace{6mm}

\item {\em Type $5$ modulo $2$ using Python syntax!}\\

\begin{enumerate}
\item[A1] {\tt 5 / 2}
\item[A2] {\tt 5 \% 2}
\item[A3] {\tt 5 \& 2}
\item[A4] {\tt 5 \&\& 2}
\end{enumerate}

\vspace{6mm}

\item {\em What will be the result of {\tt 1**4*2} ?}\\

\begin{enumerate}
\item[A1] {\tt 0}
\item[A2] {\tt 1}
\item[A3] {\tt 2}
\item[A4] {\tt 4}
\end{enumerate}

\vspace{6mm}

\item {\em What will be the result of {\tt 6 / 3*2} ?}\\

\begin{enumerate}
\item[A1] {\tt 1}
\item[A2] {\tt 0}
\item[A3] {\tt 4}
\item[A4] {\tt 1}
\end{enumerate}

\vspace{6mm}

\item {\em What will be the result of {\tt 12 / 4 / 3} ?}\\

\begin{enumerate}
\item[A1] {\tt 1}
\item[A2] {\tt 12}
\item[A3] {\tt 4}
\item[A4] {\tt 3}
\end{enumerate}

\vspace{6mm}

\item {\em What is the correct way to import all functionality from the Numpy library ?}\\

\begin{enumerate}
\item[A1] 
\begin{Verbatim}[commandchars=\\\{\}]
\PY{k+kn}{from} \PY{n+nn}{numpy} \PY{k+kn}{import} \PY{n}{all}
\end{Verbatim}
\item[A2] 
\begin{Verbatim}[commandchars=\\\{\}]
\PY{k+kn}{from} \PY{n+nn}{numpy} \PY{k+kn}{import} \PY{n}{everything}
\end{Verbatim}
\item[A3]
\begin{Verbatim}[commandchars=\\\{\}]
\PY{k+kn}{from} \PY{n+nn}{numpy} \PY{k+kn}{import} \PY{n}{*}
\end{Verbatim}
\item[A4] 
\begin{Verbatim}[commandchars=\\\{\}]
\PY{k+kn}{from} \PY{n+nn}{numpy} \PY{k+kn}{import} \PY{n}{functions}
\end{Verbatim}
\end{enumerate}

\vspace{6mm}

\item {\em Which code will calculate sin($\pi/4$)?}\\

\begin{enumerate}
\item[A1] 
\begin{Verbatim}[commandchars=\\\{\}]
\PY{k+kn}{from} \PY{n+nn}{numpy} \PY{k+kn}{import} \PY{n}{sin}
\PY{n}{sin}\PY{p}{(}\PY{n}{pi}\PY{o}{/}\PY{l+m+mi}{4}\PY{p}{)}
\end{Verbatim}
\item[A2] 
\begin{Verbatim}[commandchars=\\\{\}]
\PY{k+kn}{from} \PY{n+nn}{numpy} \PY{k+kn}{import} \PY{n}{sin}\PY{p}{,} \PY{n}{pi}
\PY{n}{sin}\PY{p}{(}\PY{n}{pi}\PY{o}{/}\PY{l+m+mi}{4}\PY{p}{)}
\end{Verbatim}
\item[A3] 
\begin{Verbatim}[commandchars=\\\{\}]
\PY{k+kn}{from} \PY{n+nn}{trigonometry} \PY{k+kn}{import} \PY{o}{sin}
\PY{n}{sin}\PY{p}{(}\PY{n}{pi}\PY{o}{/}\PY{l+m+mi}{4}\PY{p}{)}
\end{Verbatim}
\item[A4] 
\begin{Verbatim}[commandchars=\\\{\}]
\PY{k+kn}{from} \PY{n+nn}{trigonometry} \PY{k+kn}{import} \PY{n}{sin}\PY{p}{,} \PY{n}{pi}
\PY{n}{sin}\PY{p}{(}\PY{n}{pi}\PY{o}{/}\PY{l+m+mi}{4}\PY{p}{)}
\end{Verbatim}
\end{enumerate}

\vspace{6mm}

\item {\em What is the correct way to use the {\tt fractions} library to add the fractions 
$5/27 + 5/9 + 21/81$?}\\

\begin{enumerate}
\item[A1] 
\begin{Verbatim}[commandchars=\\\{\}]
\PY{k+kn}{from} \PY{n+nn}{fractions} \PY{k+kn}{import} \PY{n}{Fraction}
\PY{n}{Fraction}\PY{p}{(}\PY{l+m+mi}{5}\PY{o}{/}\PY{l+m+mi}{27}\PY{p}{)} \PY{o}{+} \PY{n}{Fraction}\PY{p}{(}\PY{l+m+mi}{5}\PY{o}{/}\PY{l+m+mi}{9}\PY{p}{)} \PY{o}{+} \PY{n}{Fraction}\PY{p}{(}\PY{l+m+mi}{21}\PY{o}{/}\PY{l+m+mi}{81}\PY{p}{)}
\end{Verbatim}
\item[A2]
\begin{Verbatim}[commandchars=\\\{\}]
\PY{k+kn}{from} \PY{n+nn}{fractions} \PY{k+kn}{import} \PY{n}{Fraction}
\PY{n}{Fraction}\PY{p}{(}\PY{l+m+mi}{5}\PY{o}{/}\PY{l+m+mi}{27} \PY{o}{+} \PY{l+m+mi}{5}\PY{o}{/}\PY{l+m+mi}{9} \PY{o}{+} \PY{l+m+mi}{21}\PY{o}{/}\PY{l+m+mi}{81}\PY{p}{)}
\end{Verbatim}
\item[A3]
\begin{Verbatim}[commandchars=\\\{\}]
\PY{k+kn}{from} \PY{n+nn}{fractions} \PY{k+kn}{import} \PY{n}{Fraction}
\PY{n}{Fraction}\PY{p}{(}\PY{l+m+mi}{5}\PY{o}{/}\PY{l+m+mf}{27.} \PY{o}{+} \PY{l+m+mi}{5}\PY{o}{/}\PY{l+m+mf}{9.} \PY{o}{+} \PY{l+m+mi}{21}\PY{o}{/}\PY{l+m+mf}{81.}\PY{p}{)}
\end{Verbatim}
\item[A4]
\begin{Verbatim}[commandchars=\\\{\}]
\PY{k+kn}{from} \PY{n+nn}{fractions} \PY{k+kn}{import} \PY{n}{Fraction}
\PY{n}{Fraction}\PY{p}{(}\PY{l+m+mi}{5}\PY{p}{,} \PY{l+m+mi}{27}\PY{p}{)} \PY{o}{+} \PY{n}{Fraction}\PY{p}{(}\PY{l+m+mi}{5}\PY{p}{,} \PY{l+m+mi}{9}\PY{p}{)} \PY{o}{+} \PY{n}{Fraction}\PY{p}{(}\PY{l+m+mi}{21}\PY{p}{,} \PY{l+m+mi}{81}\PY{p}{)}
\end{Verbatim}
\end{enumerate}

\vspace{6mm}

\item {\em Which of the following codes will calculate the greatest common divisor 
of the numbers $1377$ and $4131$ ?}\\

\begin{enumerate}
\item[A1] 
\begin{Verbatim}[commandchars=\\\{\}]
\PY{k+kn}{from} \PY{n+nn}{fractions} \PY{k+kn}{import} \PY{n}{gcd}
\PY{n}{gcd}\PY{p}{(}\PY{l+m+mi}{1377}\PY{p}{,} \PY{l+m+mi}{4131}\PY{p}{)}
\end{Verbatim}
\item[A2]
\begin{Verbatim}[commandchars=\\\{\}]
\PY{k+kn}{from} \PY{n+nn}{fractions} \PY{k+kn}{import} \PY{n}{Fraction, gcd}
\PY{n}{gcd}\PY{p}{(}\PY{l+m+mi}{1377}\PY{p}{,} \PY{l+m+mi}{4131}\PY{p}{)}
\end{Verbatim}
\item[A3]
\begin{Verbatim}[commandchars=\\\{\}]
\PY{k+kn}{from} \PY{n+nn}{fractions} \PY{k+kn}{import} \PY{n}{*}
\PY{n}{gcd}\PY{p}{(}\PY{l+m+mi}{1377}\PY{p}{,} \PY{l+m+mi}{4131}\PY{p}{)}
\end{Verbatim}
\item[A4]
\begin{Verbatim}[commandchars=\\\{\}]
\PY{k+kn}{from} \PY{n+nn}{fractions} \PY{k+kn}{import} \PY{n}{*}
\PY{n}{gcd}\PY{p}{(}\PY{l+m+mi}{4131}\PY{p}{,} \PY{l+m+mi}{1377}\PY{p}{)}
\end{Verbatim}
\end{enumerate}

\vspace{6mm}

\item {\em What will be the result of the following code?}\\

\begin{Verbatim}[commandchars=\\\{\}]
\PY{k+kn}{from} \PY{n+nn}{fractions} \PY{k+kn}{import} \PY{n}{gcd}
\PY{n}{gcd}\PY{p}{(}\PY{l+m+mi}{20}\PY{p}{,} \PY{l+m+mi}{25}\PY{p}{,} \PY{l+m+mi}{55}\PY{p}{)}
\end{Verbatim}
\vspace{6mm}

\begin{enumerate}
\item[A1] {\tt 5}
\item[A2] {\tt 20}
\item[A3] {\tt 25}
\item[A4] Error message
\end{enumerate}

\vspace{6mm}

\item {\em Which code will generate a random real number between 0 and 1 ? }\\

\begin{enumerate}
\item[A1] 
\begin{Verbatim}[commandchars=\\\{\}]
\PY{k+kn}{from} \PY{n+nn}{random} \PY{k+kn}{import} \PY{n}{random}
\PY{n}{random}\PY{p}{(0, 1}\PY{p}{)}
\end{Verbatim}
\item[A2]
\begin{Verbatim}[commandchars=\\\{\}]
\PY{k+kn}{from} \PY{n+nn}{random} \PY{k+kn}{import} \PY{n}{random}
\PY{n}{random}\PY{p}{(}\PY{p}{)}
\end{Verbatim}
\item[A3]
\begin{Verbatim}[commandchars=\\\{\}]
\PY{k+kn}{from} \PY{n+nn}{random} \PY{k+kn}{import} \PY{n}{random}
\PY{n}{random}\PY{p}{(}\PY{p}{1.0)}
\end{Verbatim}
\item[A4]
\begin{Verbatim}[commandchars=\\\{\}]
\PY{k+kn}{from} \PY{n+nn}{random} \PY{k+kn}{import} \PY{n}{rand}
\PY{n}{rand}\PY{p}{(}\PY{p}{)}
\end{Verbatim}
\end{enumerate}

\vspace{6mm}

\item {\em With the {\tt random()} function imported, which code will generate 
a random integer number between 5 and 10 ? }\\

\begin{enumerate}
\item[A1] 
\begin{Verbatim}[commandchars=\\\{\}]
\PY{l+m+mi}{5} \PY{o}{+} \PY{n+nb}{int}\PY{p}{(}\PY{l+m+mi}{5} \PY{o}{*} \PY{n}{random}\PY{p}{(}\PY{p}{)}\PY{p}{)}
\end{Verbatim}
\item[A2]
\begin{Verbatim}[commandchars=\\\{\}]
\PY{l+m+mi}{5} \PY{o}{+} \PY{l+m+mi}{5} \PY{o}{*} \PY{n+nb}{int}\PY{p}{(}\PY{n}{random}\PY{p}{(}\PY{p}{)}\PY{p}{)}
\end{Verbatim}
\item[A3]
\begin{Verbatim}[commandchars=\\\{\}]
\PY{n+nb}{int}\PY{p}{(}\PY{l+m+mi}{5} \PY{o}{+} \PY{l+m+mi}{5} \PY{o}{*} \PY{n}{random}\PY{p}{(}\PY{p}{)}\PY{p}{)}
\end{Verbatim}
\item[A4]
\begin{Verbatim}[commandchars=\\\{\}]
\PY{n+nb}{int}\PY{p}{(}\PY{l+m+mi}{10} \PY{o}{*} \PY{n}{random}\PY{p}{(}\PY{p}{)}\PY{p}{)} \PY{o}{-} \PY{l+m+mi}{5}
\end{Verbatim}
\end{enumerate}

\vspace{6mm}

\item {\em Which of the following is / are correct way(s) to define a complex number?}\\

\begin{enumerate}
\item[A1] {\tt 2 + 3j}
\item[A2] {\tt 2 + 3J}
\item[A3] {\tt 2 + 3*j}
\item[A4] {\tt 2 + 3*J}
\end{enumerate}

\vspace{6mm}

\end{enumerate}

%%%%%%%%%%%%%%%%%%%%%%%%%%%%%%%%%%%%%%%%%%%%%%%%%%%%%%%%%%%%%%%%%%%%%

\section{Functions}

\begin{enumerate}

\item {\em Why do we define custom functions in programming? }\\

\begin{enumerate}
\item[A1] To isolate self-contained functionality and make it easily reusable.
\item[A2] To make computer programs run faster.
\item[A3] To split long programs into multiple segments that have fewer lines.
\item[A4] We should not use functions, it is not a good programming practice.
\end{enumerate}

\vspace{6mm}

\item {\em When does a Python function have to return a value? }\\

\begin{enumerate}
\item[A1] When it accepts arguments.
\item[A2] When it accepts default arguments.
\item[A3] When it is used together with a print statement.
\item[A4] Never.
\end{enumerate}

\vspace{6mm}

\item {\em Which of the following codes are syntactically correct?}\\

\begin{enumerate}
\item[A1] 
\begin{Verbatim}[commandchars=\\\{\}]
\PY{k}{def} \PY{n+nf}{subtract}\PY{p}{(}\PY{n}{a}\PY{p}{,} \PY{n}{b}\PY{p}{)}
    \PY{k}{return} \PY{n}{a} \PY{o}{-} \PY{n}{b}
\end{Verbatim}
\item[A2] 
\begin{Verbatim}[commandchars=\\\{\}]
\PY{k}{def} \PY{n+nf}{subtract}\PY{p}{(}\PY{n}{a}\PY{p}{,} \PY{n}{b}\PY{p}{)}\PY{p}{:}
    \PY{k}{return} \PY{n}{a} \PY{o}{-} \PY{n}{b}
\end{Verbatim}
\item[A3] 
\begin{Verbatim}[commandchars=\\\{\}]
\PY{k}{def} \PY{n+nf}{subtract}\PY{p}{[}\PY{n}{a}\PY{p}{,} \PY{n}{b}\PY{p}{]}
    \PY{k}{return} \PY{n}{a} \PY{o}{-} \PY{n}{b}
\end{Verbatim}
\item[A4] 
\begin{Verbatim}[commandchars=\\\{\}]
\PY{k}{def} \PY{n+nf}{subtract}\PY{p}{(}\PY{n}{a}\PY{p}{,} \PY{n}{b} \PY{o}{=} \PY{l+m+mi}{5}\PY{p}{)}\PY{p}{:}
    \PY{k}{return} \PY{n}{a} \PY{o}{-} \PY{n}{b}
\end{Verbatim}
\end{enumerate}

\vspace{6mm}

\item {\em When will the function {\tt add(a, b)} defined below accept text strings 
as arguments without throwing an error?}\\

\begin{Verbatim}[commandchars=\\\{\}]
\PY{k}{def} \PY{n+nf}{add}\PY{p}{(}\PY{n}{a}\PY{p}{,} \PY{n}{b}\PY{p}{)}\PY{p}{:}
    \PY{k}{return} \PY{n}{a} \PY{o}{+} \PY{n}{b}
\end{Verbatim}
\vspace{6mm}

\begin{enumerate}
\item[A1] Only when the two text strings have the same length. 
\item[A2] Always.
\item[A3] Only when the two text strings have different lengths.
\item[A4] Never.
\end{enumerate}

\vspace{6mm}

\item {\em When do we have to specify argument types in Python functions?}\\

\begin{enumerate}
\item[A1] Never.
\item[A2] Always.
\item[A3] Always except for numbers and text strings.
\item[A4] Only when default arguments are used.
\end{enumerate}

\vspace{6mm}

\item {\em We need a function {\tt f(X)} that accepts a number {\tt X} and returns 
three values: {\tt 3*X}, {\tt 5*X}, {\tt 7*X}. Which of the following functions will do that?}\\

\begin{enumerate}
\item[A1] 
\begin{Verbatim}[commandchars=\\\{\}]
\PY{k}{def} \PY{n+nf}{f}\PY{p}{(}\PY{n}{X}\PY{p}{)}\PY{p}{:}
    \PY{k}{return} \PY{l+m+mi}{3}\PY{o}{*}\PY{n}{X} 
    \PY{k}{return} \PY{l+m+mi}{5}\PY{o}{*}\PY{n}{X} 
    \PY{k}{return} \PY{l+m+mi}{7}\PY{o}{*}\PY{n}{X}
\end{Verbatim}
\item[A2] 
\begin{Verbatim}[commandchars=\\\{\}]
\PY{k}{def} \PY{n+nf}{f}\PY{p}{(}\PY{n}{X}\PY{p}{)}\PY{p}{:}
    \PY{k}{return} \PY{l+m+mi}{3}\PY{o}{*}\PY{n}{X}\PY{p}{,} \PY{l+m+mi}{5}\PY{o}{*}\PY{n}{X}\PY{p}{,} \PY{l+m+mi}{7}\PY{o}{*}\PY{n}{X}
\end{Verbatim}
\item[A3] 
\begin{Verbatim}[commandchars=\\\{\}]
\PY{k}{def} \PY{n+nf}{f}\PY{p}{(}\PY{n}{X}\PY{p}{)}\PY{p}{:}
    \PY{k}{return} \PY{l+m+mi}{3}\PY{o}{*}\PY{n}{X} \PY{o}{+} \PY{l+m+mi}{5}\PY{o}{*}\PY{n}{X} \PY{o}{+} \PY{l+m+mi}{7}\PY{o}{*}\PY{n}{X}
\end{Verbatim}
\item[A4] 
\begin{Verbatim}[commandchars=\\\{\}]
\PY{k}{def} \PY{n+nf}{f}\PY{p}{(}\PY{n}{X}\PY{p}{)}\PY{p}{:}
    \PY{l+m+mi}{3}\PY{o}{*}\PY{n}{X}\PY{p}{,} \PY{l+m+mi}{5}\PY{o}{*}\PY{n}{X}\PY{p}{,} \PY{l+m+mi}{7}\PY{o}{*}\PY{n}{X}
    \PY{k}{return} 
\end{Verbatim}
\end{enumerate}

\vspace{6mm}

\item {\em Which of the following four function definitions are syntactically correct?}\\

\begin{enumerate}
\item[A1] 
\begin{Verbatim}[commandchars=\\\{\}]
\PY{k+kn}{from} \PY{n+nn}{numpy} \PY{k+kn}{import} \PY{n}{sqrt}
\PY{k}{def} \PY{n+nf}{hypotenuse}\PY{p}{(}\PY{n}{a}\PY{p}{,} \PY{n}{b}\PY{p}{)}\PY{p}{:}
    \PY{k}{return} \PY{n}{sqrt}\PY{p}{(}\PY{n}{a}\PY{o}{*}\PY{o}{*}\PY{l+m+mi}{2} \PY{o}{+} \PY{n}{b}\PY{o}{*}\PY{o}{*}\PY{l+m+mi}{2}\PY{p}{)}
\end{Verbatim}
\item[A2] 
\begin{Verbatim}[commandchars=\\\{\}]
\PY{k+kn}{from} \PY{n+nn}{numpy} \PY{k+kn}{import} \PY{n}{sqrt}
\PY{k}{def} \PY{n+nf}{hypotenuse}\PY{p}{(}\PY{n}{a}\PY{o}{=}\PY{l+m+mi}{5}\PY{p}{,} \PY{n}{b}\PY{p}{)}\PY{p}{:}
    \PY{k}{return} \PY{n}{sqrt}\PY{p}{(}\PY{n}{a}\PY{o}{*}\PY{o}{*}\PY{l+m+mi}{2} \PY{o}{+} \PY{n}{b}\PY{o}{*}\PY{o}{*}\PY{l+m+mi}{2}\PY{p}{)}
\end{Verbatim}
\item[A3] 
\begin{Verbatim}[commandchars=\\\{\}]
\PY{k+kn}{from} \PY{n+nn}{numpy} \PY{k+kn}{import} \PY{n}{sqrt}
\PY{k}{def} \PY{n+nf}{hypotenuse}\PY{p}{(}\PY{n}{a}\PY{p}{,} \PY{n}{b}\PY{o}{=}\PY{l+m+mi}{5}\PY{p}{)}\PY{p}{:}
    \PY{k}{return} \PY{n}{sqrt}\PY{p}{(}\PY{n}{a}\PY{o}{*}\PY{o}{*}\PY{l+m+mi}{2} \PY{o}{+} \PY{n}{b}\PY{o}{*}\PY{o}{*}\PY{l+m+mi}{2}\PY{p}{)}
\end{Verbatim}
\item[A4] 
\begin{Verbatim}[commandchars=\\\{\}]
\PY{k+kn}{from} \PY{n+nn}{numpy} \PY{k+kn}{import} \PY{n}{sqrt}
\PY{k}{def} \PY{n+nf}{hypotenuse}\PY{p}{(}\PY{n}{a}\PY{o}{=}\PY{l+m+mi}{5}\PY{p}{,} \PY{n}{b}\PY{o}{=}\PY{l+m+mi}{5}\PY{p}{)}\PY{p}{:}
    \PY{k}{return} \PY{n}{sqrt}\PY{p}{(}\PY{n}{a}\PY{o}{*}\PY{o}{*}\PY{l+m+mi}{2} \PY{o}{+} \PY{n}{b}\PY{o}{*}\PY{o}{*}\PY{l+m+mi}{2}\PY{p}{)}
\end{Verbatim}
\end{enumerate}

\vspace{6mm}

\item {\em For the function {\tt compose} defined below, what function call(s) will not cause an error message?}\\

\begin{Verbatim}[commandchars=\\\{\}]
\PY{k}{def} \PY{n+nf}{compose}\PY{p}{(}\PY{n}{a}\PY{p}{,} \PY{n}{x} \PY{o}{=} \PY{l+m+mi}{1}\PY{p}{,} \PY{n}{y} \PY{o}{=} \PY{l+m+mi}{2}\PY{p}{,} \PY{n}{z} \PY{o}{=} \PY{l+m+mi}{3}\PY{p}{)}\PY{p}{:}
    \PY{k}{return} \PY{n}{a} \PY{o}{+} \PY{n}{x} \PY{o}{+} \PY{n}{y} \PY{o}{+} \PY{n}{z}
\end{Verbatim}
\vspace{6mm}

\begin{enumerate}
\item[A1] 
\begin{Verbatim}[commandchars=\\\{\}]
\PY{k}{print} \PY{n}{compose}\PY{p}{(}\PY{l+m+mi}{1}\PY{p}{)}
\end{Verbatim}
\item[A2] 
\begin{Verbatim}[commandchars=\\\{\}]
\PY{k}{print} \PY{n}{compose}\PY{p}{(}\PY{l+m+mi}{1}\PY{p}{,} \PY{n}{y} \PY{o}{=} \PY{l+m+mi}{10}\PY{p}{)}
\end{Verbatim}
\item[A3] 
\begin{Verbatim}[commandchars=\\\{\}]
\PY{k}{print} \PY{n}{compose}\PY{p}{(}\PY{l+m+mi}{1}\PY{p}{,} \PY{n}{x} \PY{o}{=} \PY{l+m+mi}{20}\PY{p}{,} \PY{n}{z} \PY{o}{=} \PY{l+m+mi}{5}\PY{p}{)}
\end{Verbatim}
\item[A4] 
\begin{Verbatim}[commandchars=\\\{\}]
\PY{k}{print} \PY{n}{compose}\PY{p}{(}\PY{l+m+mi}{1}\PY{p}{,} \PY{l+m+mi}{10}\PY{p}{,} \PY{n}{x} \PY{o}{=} \PY{l+m+mi}{20}\PY{p}{,} \PY{n}{z} \PY{o}{=} \PY{l+m+mi}{5}\PY{p}{)}
\end{Verbatim}
\end{enumerate}

\vspace{6mm}

\end{enumerate}


%%%%%%%%%%%%%%%%%%%%%%%%%%%%%%%%%%%%%%%%%%%%%%%%%%%%%%%%%%%%%%%%%%%%%

\section{Colors and plotting}

\begin{enumerate}

\item {\em Which of the following RGB values define(s) a green color?}\\

\begin{enumerate}
\item[A1] [1, 0, 0]
\item[A2] [1, 0.5, 0]
\item[A3] [0, 0.5, 0]
\item[A4] [0, 0.5, 1]
\end{enumerate}

\vspace{6mm}

\item {\em Which of the following RGB values define(s) a shade of grey?}\\

\begin{enumerate}
\item[A1] [0.2, 0.2, 0.2]
\item[A2] [0.5, 0.6, 0.7]
\item[A3] [1, 0.5, 0]
\item[A4] [1, 0.5, 1]
\end{enumerate}

\vspace{6mm}

\item {\em Which of the following RGB values define(s) color that is closest to purple?}\\

\begin{enumerate}
\item[A1] [1, 0.8, 0]
\item[A2] [0.5, 0.6, 0.1]
\item[A3] [0.5, 0, 0.5]
\item[A4] [1, 0.5, 1]
\end{enumerate}

\vspace{6mm}

\item {\em Which of the following codes will draw a square with vertices 
[0, 0], [1, 0], [1, 1], [0, 1]?}\\

\begin{enumerate}
\item[A1] 
\begin{Verbatim}[commandchars=\\\{\}]
\PY{k+kn}{from} \PY{n+nn}{pylab} \PY{k+kn}{import} \PY{o}{*}
\PY{n}{x} \PY{o}{=} \PY{p}{[}\PY{l+m+mf}{0.0}\PY{p}{,} \PY{l+m+mf}{1.0}\PY{p}{,} \PY{l+m+mf}{1.0}\PY{p}{,} \PY{l+m+mf}{0.0}\PY{p}{]}
\PY{n}{y} \PY{o}{=} \PY{p}{[}\PY{l+m+mf}{0.0}\PY{p}{,} \PY{l+m+mf}{0.0}\PY{p}{,} \PY{l+m+mf}{1.0}\PY{p}{,} \PY{l+m+mf}{1.0}\PY{p}{]}
\PY{n}{clf}\PY{p}{(}\PY{p}{)}
\PY{n}{plot}\PY{p}{(}\PY{n}{x}\PY{p}{,} \PY{n}{y}\PY{p}{)}
\PY{n}{lab}\PY{o}{.}\PY{n}{show}\PY{p}{(}\PY{p}{)}
\end{Verbatim}
\item[A2] 
\begin{Verbatim}[commandchars=\\\{\}]
\PY{k+kn}{from} \PY{n+nn}{pylab} \PY{k+kn}{import} \PY{o}{*}
\PY{n}{x} \PY{o}{=} \PY{p}{[}\PY{l+m+mf}{0.0}\PY{p}{,} \PY{l+m+mf}{0.0}\PY{p}{,} \PY{l+m+mf}{1.0}\PY{p}{,} \PY{l+m+mf}{1.0}\PY{p}{]}
\PY{n}{y} \PY{o}{=} \PY{p}{[}\PY{l+m+mf}{0.0}\PY{p}{,} \PY{l+m+mf}{1.0}\PY{p}{,} \PY{l+m+mf}{1.0}\PY{p}{,} \PY{l+m+mf}{0.0}\PY{p}{]}
\PY{n}{clf}\PY{p}{(}\PY{p}{)}
\PY{n}{plot}\PY{p}{(}\PY{n}{x}\PY{p}{,} \PY{n}{y}\PY{p}{)}
\PY{n}{lab}\PY{o}{.}\PY{n}{show}\PY{p}{(}\PY{p}{)}
\end{Verbatim}
\item[A3] 
\begin{Verbatim}[commandchars=\\\{\}]
\PY{k+kn}{from} \PY{n+nn}{pylab} \PY{k+kn}{import} \PY{o}{*}
\PY{n}{x} \PY{o}{=} \PY{p}{[}\PY{l+m+mf}{0.0}\PY{p}{,} \PY{l+m+mf}{1.0}\PY{p}{,} \PY{l+m+mf}{1.0}\PY{p}{,} \PY{l+m+mf}{0.0}\PY{p}{,} \PY{l+m+mf}{0.0}\PY{p}{]}
\PY{n}{y} \PY{o}{=} \PY{p}{[}\PY{l+m+mf}{0.0}\PY{p}{,} \PY{l+m+mf}{0.0}\PY{p}{,} \PY{l+m+mf}{1.0}\PY{p}{,} \PY{l+m+mf}{1.0}\PY{p}{,} \PY{l+m+mf}{0.0}\PY{p}{]}
\PY{n}{clf}\PY{p}{(}\PY{p}{)}
\PY{n}{plot}\PY{p}{(}\PY{n}{x}\PY{p}{,} \PY{n}{y}\PY{p}{)}
\PY{n}{lab}\PY{o}{.}\PY{n}{show}\PY{p}{(}\PY{p}{)}
\end{Verbatim}
\item[A4] 
\begin{Verbatim}[commandchars=\\\{\}]
\PY{k+kn}{from} \PY{n+nn}{pylab} \PY{k+kn}{import} \PY{o}{*}
\PY{n}{x} \PY{o}{=} \PY{p}{[}\PY{l+m+mf}{0.0}\PY{p}{,} \PY{l+m+mf}{0.0}\PY{p}{,} \PY{l+m+mf}{1.0}\PY{p}{,} \PY{l+m+mf}{1.0}\PY{p}{,} \PY{l+m+mf}{0.0}\PY{p}{]}
\PY{n}{y} \PY{o}{=} \PY{p}{[}\PY{l+m+mf}{0.0}\PY{p}{,} \PY{l+m+mf}{1.0}\PY{p}{,} \PY{l+m+mf}{1.0}\PY{p}{,} \PY{l+m+mf}{0.0}\PY{p}{,} \PY{l+m+mf}{1.0}\PY{p}{]}
\PY{n}{clf}\PY{p}{(}\PY{p}{)}
\PY{n}{plot}\PY{p}{(}\PY{n}{x}\PY{p}{,} \PY{n}{y}\PY{p}{)}
\PY{n}{lab}\PY{o}{.}\PY{n}{show}\PY{p}{(}\PY{p}{)}
\end{Verbatim}
\end{enumerate}

\vspace{6mm}

\item {\em Which of the following codes will define an array {\tt X} of 20 equidistant points 
covering the interval [0, 10]?}\\

\begin{enumerate}
\item[A1]
\begin{Verbatim}[commandchars=\\\{\}]
\PY{k+kn}{from} \PY{n+nn}{numpy} \PY{k+kn}{import} \PY{n}{linspace}
\PY{n}{X} \PY{o}{=} \PY{n}{linspace}\PY{p}{(}\PY{l+m+mi}{0}\PY{p}{,} \PY{l+m+mi}{10}\PY{p}{,} \PY{l+m+mi}{20}\PY{p}{)}
\end{Verbatim}
\item[A2]
\begin{Verbatim}[commandchars=\\\{\}]
\PY{k+kn}{from} \PY{n+nn}{numpy} \PY{k+kn}{import} \PY{n}{linspace}
\PY{n}{X} \PY{o}{=} \PY{n}{linspace}\PY{p}{(}\PY{l+m+mi}{0}\PY{p}{,} \PY{l+m+mi}{10}\PY{p}{,} \PY{l+m+mi}{0.5}\PY{p}{)}
\end{Verbatim}
\item[A3]
\begin{Verbatim}[commandchars=\\\{\}]
\PY{k+kn}{from} \PY{n+nn}{numpy} \PY{k+kn}{import} \PY{n}{linspace}
\PY{n}{X} \PY{o}{=} \PY{n}{linspace}\PY{p}{(}\PY{l+m+mi}{0} \PY{p}{:} \PY{l+m+mi}{10} \PY{p}{:} \PY{l+m+mi}{20}\PY{p}{)}
\end{Verbatim}
\item[A4]
\begin{Verbatim}[commandchars=\\\{\}]
\PY{k+kn}{from} \PY{n+nn}{numpy} \PY{k+kn}{import} \PY{n}{linspace}
\PY{n}{X} \PY{o}{=} \PY{n}{linspace}\PY{p}{(}\PY{l+m+mi}{0} \PY{p}{:} \PY{l+m+mi}{10} \PY{p}{:} \PY{l+m+mf}{0.5}\PY{p}{)}
\end{Verbatim}
\end{enumerate}

\vspace{6mm}

\item {\em What will be the output of the following code, each value 
rounded to an integer value?}\\

\begin{Verbatim}[commandchars=\\\{\}]
\PY{k+kn}{from} \PY{n+nn}{numpy} \PY{k+kn}{import} \PY{n}{linspace, sin, pi}
\PY{n}{X} \PY{o}{=} \PY{n}{linspace}\PY{p}{(}\PY{l+m+mi}{0}\PY{p}{,} \PY{l+m+mi}{3}\PY{o}{*}\PY{n}{pi}\PY{p}{,} \PY{l+m+mi}{4}\PY{p}{)}
\PY{n}{Y} \PY{o}{=} \PY{n}{sin}\PY{p}{(}\PY{n}{X}\PY{p}{)}
\PY{k}{print} \PY{n}{Y}
\end{Verbatim}
\vspace{6mm}

\begin{enumerate}
\item[A1] {\tt [0  0  0]}
\item[A2] {\tt [0  0  0  0]}
\item[A3] {\tt [0  1  0  -1  0]}
\item[A4] {\tt [0  0  0  0  0]}
\end{enumerate}

\vspace{6mm}

\item {\em What is the meaning of the Pylab {\tt clf()} function?}\\

\begin{enumerate}
\item[A1] Cleans the canvas. 
\item[A2] Shows the legend.
\item[A3] Makes both axes to be equally scaled.
\item[A4] Suppresses warnings.
\end{enumerate}

\vspace{6mm}

\item {\em What Pylab function is used to enforce equal scaling of the horizontal and 
vertical axes?}\\

\begin{enumerate}
\item[A1] {\tt axes = "equal"}
\item[A2] {\tt equal('axes')}
\item[A3] {\tt axes('equal')}
\item[A4] {\tt axis('equal')}
\end{enumerate}

\vspace{6mm}

\item {\em What Pylab function is used to show the legend? Should it be used before or after 
the {\tt plot()} function?}\\

\begin{enumerate}
\item[A1] {\tt showlegend()}, to be used before the {\tt plot()} function.
\item[A2] {\tt showlegend()}, to be used after the {\tt plot()} function.
\item[A3] {\tt legend()}, to be used before the {\tt plot()} function.
\item[A4] {\tt legend()}, to be used after the {\tt plot()} function.
\end{enumerate}

\vspace{6mm}

\item {\em Which of the following codes will plot a cosine function 
in the interval $(0, \pi)$ using a dashed green line?}\\

\begin{enumerate}
\item[A1] 
\begin{Verbatim}[commandchars=\\\{\}]
\PY{k+kn}{from} \PY{n+nn}{numpy} \PY{k+kn}{import} \PY{o}{*}
\PY{k+kn}{from} \PY{n+nn}{pylab} \PY{k+kn}{import} \PY{o}{*}
\PY{n}{x} \PY{o}{=} \PY{n}{linspace}\PY{p}{(}\PY{l+m+mi}{0}\PY{p}{,} \PY{n}{pi}\PY{p}{,} \PY{l+m+mi}{100}\PY{p}{)}
\PY{n}{y} \PY{o}{=} \PY{n}{cos}\PY{p}{(}\PY{n}{x}\PY{p}{)}
\PY{n}{axis}\PY{o}{=}\PY{p}{(}\PY{l+s}{"}\PY{l+s}{equal}\PY{l+s}{"}\PY{p}{)}
\PY{n}{clf}\PY{p}{(}\PY{p}{)}
\PY{n}{plot}\PY{p}{(}\PY{n}{x}\PY{p}{,} \PY{n}{y}\PY{p}{,} \PY{l+s}{'}\PY{l+s}{g-}\PY{l+s}{'}\PY{p}{,} \PY{n}{label}\PY{o}{=}\PY{l+s}{"}\PY{l+s}{cos(x)}\PY{l+s}{"}\PY{p}{)}
\PY{n}{legend}\PY{p}{(}\PY{p}{)}
\PY{n}{lab}\PY{o}{.}\PY{n}{show}\PY{p}{(}\PY{p}{)}
\end{Verbatim}
\item[A2] 
\begin{Verbatim}[commandchars=\\\{\}]
\PY{k+kn}{from} \PY{n+nn}{numpy} \PY{k+kn}{import} \PY{o}{*}
\PY{k+kn}{from} \PY{n+nn}{pylab} \PY{k+kn}{import} \PY{o}{*}
\PY{n}{x} \PY{o}{=} \PY{n}{linspace}\PY{p}{(}\PY{l+m+mi}{0}\PY{p}{,} \PY{n}{pi}\PY{p}{,} \PY{l+m+mi}{100}\PY{p}{)}
\PY{n}{y} \PY{o}{=} \PY{n}{cos}\PY{p}{(}\PY{n}{x}\PY{p}{)}
\PY{n}{clf}\PY{p}{(}\PY{p}{)}
\PY{n}{plot}\PY{p}{(}\PY{n}{x}\PY{p}{,} \PY{n}{y}\PY{p}{,} \PY{l+s}{'}\PY{l+s}{g--}\PY{l+s}{'}\PY{p}{)}
\PY{n}{lab}\PY{o}{.}\PY{n}{show}\PY{p}{(}\PY{p}{)}
\end{Verbatim}
\item[A3] 
\begin{Verbatim}[commandchars=\\\{\}]
\PY{k+kn}{from} \PY{n+nn}{numpy} \PY{k+kn}{import} \PY{o}{*}
\PY{k+kn}{from} \PY{n+nn}{pylab} \PY{k+kn}{import} \PY{o}{*}
\PY{n}{x} \PY{o}{=} \PY{n}{linspace}\PY{p}{(}\PY{l+m+mi}{0}\PY{p}{,} \PY{n}{pi}\PY{o}{/}\PY{l+m+mi}{2}\PY{p}{,} \PY{l+m+mi}{100}\PY{p}{)}
\PY{n}{y} \PY{o}{=} \PY{n}{cos}\PY{p}{(}\PY{n}{x}\PY{p}{)}
\PY{n}{axis}\PY{o}{=}\PY{p}{(}\PY{l+s}{"}\PY{l+s}{equal}\PY{l+s}{"}\PY{p}{)}
\PY{n}{clf}\PY{p}{(}\PY{p}{)}
\PY{n}{plot}\PY{p}{(}\PY{n}{x}\PY{p}{,} \PY{n}{y}\PY{p}{,} \PY{l+s}{'}\PY{l+s}{g--}\PY{l+s}{'}\PY{p}{,} \PY{n}{label}\PY{o}{=}\PY{l+s}{"}\PY{l+s}{cos(x)}\PY{l+s}{"}\PY{p}{)}
\PY{n}{legend}\PY{p}{(}\PY{p}{)}
\PY{n}{lab}\PY{o}{.}\PY{n}{show}\PY{p}{(}\PY{p}{)}
\end{Verbatim}
\item[A4] 
\begin{Verbatim}[commandchars=\\\{\}]
\PY{k+kn}{from} \PY{n+nn}{numpy} \PY{k+kn}{import} \PY{o}{*}
\PY{k+kn}{from} \PY{n+nn}{pylab} \PY{k+kn}{import} \PY{o}{*}
\PY{n}{x} \PY{o}{=} \PY{n}{linspace}\PY{p}{(}\PY{l+m+mi}{0}\PY{p}{,} \PY{n}{pi}\PY{p}{,} \PY{l+m+mi}{100}\PY{p}{)}
\PY{n}{y} \PY{o}{=} \PY{n}{cos}\PY{p}{(}\PY{n}{x}\PY{p}{)}
\PY{n}{axis}\PY{o}{=}\PY{p}{(}\PY{l+s}{"}\PY{l+s}{equal}\PY{l+s}{"}\PY{p}{)}
\PY{n}{clf}\PY{p}{(}\PY{p}{)}
\PY{n}{plot}\PY{p}{(}\PY{n}{x}\PY{p}{,} \PY{n}{y}\PY{p}{,} \PY{l+s}{'}\PY{l+s}{b--}\PY{l+s}{'}\PY{p}{,} \PY{n}{label}\PY{o}{=}\PY{l+s}{"}\PY{l+s}{cos(x)}\PY{l+s}{"}\PY{p}{)}
\PY{n}{legend}\PY{p}{(}\PY{p}{)}
\PY{n}{lab}\PY{o}{.}\PY{n}{show}\PY{p}{(}\PY{p}{)}
\end{Verbatim}
\end{enumerate}

\vspace{6mm}

\item {\em We have two arrays {\tt x} and {\tt y} created via the {\tt linspace} function. What 
Numpy function will create a 2D Cartesian product grid of them?}\\

\begin{enumerate}
\item[A1] {\tt X, Y  = meshgrid(x, y)}
\item[A2] {\tt X, Y  = cartesiangrid(x, y)}
\item[A3] {\tt X, Y  = productgrid(x, y)}
\item[A4] {\tt X, Y  = product(x, y)}
\end{enumerate}

\vspace{6mm}

\item {\em The arrays {\tt X} and {\tt Y} represent a 2D Cartesian product grid. How can we
assign values of the function $1 / (1 + x^2 + y^2)$ to the grid points?}\\

\begin{enumerate}
\item[A1] {\tt Z = 1. / (1 + exp(X) + exp(Y))}
\item[A2] {\tt Z = 1. / (1 + X*X + Y*Y)}
\item[A3] {\tt Z = 1. / (1 + X\^{}2 + Y\^{}2)}
\item[A4] {\tt Z = 1. / (1 + X**2 + Y**2)}
\end{enumerate}

\vspace{6mm}

\item {\em Only one of the following codes will display a simple pie chart. Which one is it?}\\

\begin{enumerate}
\item[A1] 
\begin{Verbatim}[commandchars=\\\{\}]
\PY{k+kn}{from} \PY{n+nn}{pylab} \PY{k+kn}{import} \PY{o}{*}
\PY{n}{data} \PY{o}{=} \PY{p}{[}\PY{l+m+mi}{20} \PY{o}{\PYZpc{}}\PY{p}{,} \PY{l+m+mi}{40} \PY{o}{\PYZpc{}}\PY{p}{,} \PY{l+m+mi}{40} \PY{o}{\PYZpc{}}\PY{p}{]}
\PY{n}{labels} \PY{o}{=} \PY{p}{[}\PY{l+s}{'}\PY{l+s}{bronze}\PY{l+s}{'}\PY{p}{,} \PY{l+s}{'}\PY{l+s}{silver}\PY{l+s}{'}\PY{p}{,} \PY{l+s}{'}\PY{l+s}{gold}\PY{l+s}{'}\PY{p}{]}
\PY{n}{clf}\PY{p}{(}\PY{p}{)}
\PY{n}{axis}\PY{p}{(}\PY{l+s}{'}\PY{l+s}{equal}\PY{l+s}{'}\PY{p}{)}
\PY{n}{pie}\PY{p}{(}\PY{n}{data}\PY{p}{,} \PY{n}{labels}\PY{o}{=}\PY{n}{labels}\PY{p}{)}
\PY{n}{lab}\PY{o}{.}\PY{n}{show}\PY{p}{(}\PY{p}{)}
\end{Verbatim}
\item[A2] 
\begin{Verbatim}[commandchars=\\\{\}]
\PY{k+kn}{from} \PY{n+nn}{pylab} \PY{k+kn}{import} \PY{o}{*}
\PY{n}{data} \PY{o}{=} \PY{p}{[}\PY{l+s}{'}\PY{l+s}{bronze}\PY{l+s}{'}\PY{p}{,} \PY{l+s}{'}\PY{l+s}{silver}\PY{l+s}{'}\PY{p}{,} \PY{l+s}{'}\PY{l+s}{gold}\PY{l+s}{'}\PY{p}{]}
\PY{n}{labels} \PY{o}{=} \PY{p}{[}\PY{l+m+mi}{20}\PY{p}{,} \PY{l+m+mi}{40}\PY{p}{,} \PY{l+m+mi}{40}\PY{p}{]}
\PY{n}{clf}\PY{p}{(}\PY{p}{)}
\PY{n}{axis}\PY{p}{(}\PY{l+s}{'}\PY{l+s}{equal}\PY{l+s}{'}\PY{p}{)}
\PY{n}{pie}\PY{p}{(}\PY{n}{data}\PY{p}{,} \PY{n}{labels}\PY{o}{=}\PY{n}{labels}\PY{p}{)}
\PY{n}{lab}\PY{o}{.}\PY{n}{show}\PY{p}{(}\PY{p}{)}
\end{Verbatim}
\item[A3]
\begin{Verbatim}[commandchars=\\\{\}]
\PY{k+kn}{from} \PY{n+nn}{pylab} \PY{k+kn}{import} \PY{o}{*}
\PY{n}{data} \PY{o}{=} \PY{p}{[}\PY{l+m+mi}{20}\PY{p}{,} \PY{l+m+mi}{40}\PY{p}{,} \PY{l+m+mi}{40}\PY{p}{]}
\PY{n}{labels} \PY{o}{=} \PY{p}{[}\PY{l+s}{'}\PY{l+s}{bronze}\PY{l+s}{'}\PY{p}{,} \PY{l+s}{'}\PY{l+s}{silver}\PY{l+s}{'}\PY{p}{,} \PY{l+s}{'}\PY{l+s}{gold}\PY{l+s}{'}\PY{p}{]}
\PY{n}{clf}\PY{p}{(}\PY{p}{)}
\PY{n}{axis}\PY{p}{(}\PY{l+s}{'}\PY{l+s}{equal}\PY{l+s}{'}\PY{p}{)}
\PY{n}{pie}\PY{p}{(}\PY{n}{data}\PY{p}{,} \PY{n}{labels}\PY{o}{=}\PY{n}{labels}\PY{p}{)}
\PY{n}{lab}\PY{o}{.}\PY{n}{show}\PY{p}{(}\PY{p}{)}
\end{Verbatim}
\item[A4] 
\begin{Verbatim}[commandchars=\\\{\}]
\PY{k+kn}{from} \PY{n+nn}{pylab} \PY{k+kn}{import} \PY{o}{*}
\PY{n}{data} \PY{o}{=} \PY{p}{[}\PY{l+m+mi}{20}\PY{p}{,} \PY{l+m+mi}{40}\PY{p}{,} \PY{l+m+mi}{40}\PY{p}{]}
\PY{n}{labels} \PY{o}{=} \PY{p}{[}\PY{l+s}{'}\PY{l+s}{bronze}\PY{l+s}{'}\PY{p}{,} \PY{l+s}{'}\PY{l+s}{silver}\PY{l+s}{'}\PY{p}{,} \PY{l+s}{'}\PY{l+s}{gold}\PY{l+s}{'}\PY{p}{]}
\PY{n}{clf}\PY{p}{(}\PY{p}{)}
\PY{n}{axis}\PY{p}{(}\PY{l+s}{'}\PY{l+s}{equal}\PY{l+s}{'}\PY{p}{)}
\PY{n}{pie}\PY{p}{(}\PY{p}{)}
\PY{n}{lab}\PY{o}{.}\PY{n}{show}\PY{p}{(}\PY{p}{)}
\end{Verbatim}
\end{enumerate}

\vspace{6mm}

\item {\em Which of the following codes will display a simple bar chart with the values 5, 7, 3, 9 with bars
starting at 0, 2, 4, and 6 on the horizontal axis?}\\

\begin{enumerate}
\item[A1] 
\begin{Verbatim}[commandchars=\\\{\}]
\PY{k+kn}{from} \PY{n+nn}{pylab} \PY{k+kn}{import} \PY{o}{*}
\PY{n}{data} \PY{o}{=} \PY{p}{[}\PY{l+m+mi}{5}\PY{p}{,} \PY{l+m+mi}{7}\PY{p}{,} \PY{l+m+mi}{3}\PY{p}{,} \PY{l+m+mi}{9}\PY{p}{]}
\PY{n}{clf}\PY{p}{(}\PY{p}{)}
\PY{n}{bar}\PY{p}{(}\PY{p}{[}\PY{l+m+mi}{0}\PY{p}{,} \PY{l+m+mi}{2}\PY{p}{,} \PY{l+m+mi}{4}\PY{p}{,} \PY{l+m+mi}{6}\PY{p}{]}\PY{p}{,} \PY{n}{data}\PY{p}{)}
\PY{n}{lab}\PY{o}{.}\PY{n}{show}\PY{p}{(}\PY{p}{)}
\end{Verbatim}
\item[A2] 
\begin{Verbatim}[commandchars=\\\{\}]
\PY{k+kn}{from} \PY{n+nn}{pylab} \PY{k+kn}{import} \PY{o}{*}
\PY{n}{data} \PY{o}{=} \PY{p}{[}\PY{l+m+mi}{5}\PY{p}{,} \PY{l+m+mi}{7}\PY{p}{,} \PY{l+m+mi}{3}\PY{p}{,} \PY{l+m+mi}{9}\PY{p}{]}
\PY{n}{clf}\PY{p}{(}\PY{p}{)}
\PY{n}{bar_chart}\PY{p}{(}\PY{p}{[}\PY{l+m+mi}{0}\PY{p}{,} \PY{l+m+mi}{2}\PY{p}{,} \PY{l+m+mi}{4}\PY{p}{,} \PY{l+m+mi}{6}\PY{p}{]}\PY{p}{,} \PY{n}{data}\PY{p}{)}
\PY{n}{lab}\PY{o}{.}\PY{n}{show}\PY{p}{(}\PY{p}{)}
\end{Verbatim}
\item[A3]
\begin{Verbatim}[commandchars=\\\{\}]
\PY{k+kn}{from} \PY{n+nn}{pylab} \PY{k+kn}{import} \PY{o}{*}
\PY{n}{data} \PY{o}{=} \PY{p}{[}\PY{l+m+mi}{0}\PY{p}{,} \PY{l+m+mi}{2}\PY{p}{,} \PY{l+m+mi}{4}\PY{p}{,} \PY{l+m+mi}{6}\PY{p}{]}
\PY{n}{clf}\PY{p}{(}\PY{p}{)}
\PY{n}{bar}\PY{p}{(}\PY{p}{[}\PY{l+m+mi}{5}\PY{p}{,} \PY{l+m+mi}{7}\PY{p}{,} \PY{l+m+mi}{3}\PY{p}{,} \PY{l+m+mi}{9}\PY{p}{]}\PY{p}{,} \PY{n}{data}\PY{p}{)}
\PY{n}{lab}\PY{o}{.}\PY{n}{show}\PY{p}{(}\PY{p}{)}
\end{Verbatim}
\item[A4] 
\begin{Verbatim}[commandchars=\\\{\}]
\PY{k+kn}{from} \PY{n+nn}{pylab} \PY{k+kn}{import} \PY{o}{*}
\PY{n}{data} \PY{o}{=} \PY{p}{[}\PY{l+m+mi}{0}\PY{p}{,} \PY{l+m+mi}{2}\PY{p}{,} \PY{l+m+mi}{4}\PY{p}{,} \PY{l+m+mi}{6}\PY{p}{]}
\PY{n}{clf}\PY{p}{(}\PY{p}{)}
\PY{n}{bar\PYZus{}chart}\PY{p}{(}\PY{p}{[}\PY{l+m+mi}{5}\PY{p}{,} \PY{l+m+mi}{7}\PY{p}{,} \PY{l+m+mi}{3}\PY{p}{,} \PY{l+m+mi}{9}\PY{p}{]}\PY{p}{,} \PY{n}{data}\PY{p}{)}
\PY{n}{lab}\PY{o}{.}\PY{n}{show}\PY{p}{(}\PY{p}{)}
\end{Verbatim}
\end{enumerate}

\vspace{6mm}

\end{enumerate}

%%%%%%%%%%%%%%%%%%%%%%%%%%%%%%%%%%%%%%%%%%%%%%%%%%%%%%%%%%%%%%%

\section{Variables}

\begin{enumerate}
\item {\em What are correct ways to assign the value {\tt 2.0} to a variable {\tt var}?}\\

\begin{enumerate}
\item[A1] 
\begin{verbatim}
val := 2.0
\end{verbatim}
\item[A2] 
\begin{verbatim}
val == 2.0
\end{verbatim}
\item[A3] 
\begin{verbatim}
val(2.0)
\end{verbatim}
\item[A4] 
\begin{verbatim}
val = 2.0
\end{verbatim}
\end{enumerate}

\vspace{6mm}

\item {\em What of the following are correct ways to store the text string ``{\tt Pontiac}'' in a variable {\tt car}?}\\

\begin{enumerate}
\item[A1] 
\begin{verbatim}
car = Pontiac
\end{verbatim}
\item[A2] 
\begin{verbatim}
car = "Pontiac"
\end{verbatim}
\item[A3] 
\begin{verbatim}
car = 'Pontiac'
\end{verbatim}
\item[A4] 
\begin{verbatim}
car := "Pontiac"
\end{verbatim}
\end{enumerate}

\vspace{6mm}

\item {\em What is the correct way to create a Boolean variable {\tt answer} whose value is {\tt True}?}\\

\begin{enumerate}
\item[A1] 
\begin{verbatim}
answer = True
\end{verbatim}
\item[A2] 
\begin{verbatim}
answer = 'True'
\end{verbatim}
\item[A3] 
\begin{verbatim}
answer = 1
\end{verbatim}
\item[A4] 
\begin{verbatim}
answer = "True"
\end{verbatim}
\end{enumerate}

\vspace{6mm}

\item {\em None of the variables {\tt a}, {\tt b} was declared before. 
Which of the following codes are correct?}\\

\begin{enumerate}
\item[A1] 
\begin{verbatim}
a = b = 1
\end{verbatim}
\item[A2] 
\begin{verbatim}
a = 1 = b
\end{verbatim}
\item[A3] 
\begin{verbatim}
a = a
\end{verbatim}
\item[A4] 
\begin{verbatim}
a + b = 5
\end{verbatim}
\end{enumerate}

\vspace{6mm}

\item {\em What of the following is the correct way to increase the value of an existing integer 
variable {\tt val} by 10 and print the result?}\\

\begin{enumerate}
\item[A1] 
\begin{verbatim}
val + 10
\end{verbatim}
\item[A2] 
\begin{verbatim}
print val + 10
\end{verbatim}
\item[A3] 
\begin{verbatim}
print val += 10
\end{verbatim}
\item[A4] 
\begin{verbatim}
val += 10
print val
\end{verbatim}
\end{enumerate}

\vspace{6mm}

\item {\em When can a given variable have different types in various parts 
of a Python program?}\\

\begin{enumerate}
\item[A1] Any time.
\item[A2] Never.
\item[A3] Only when variable shadowing takes place.
\item[A4] Only when the types are real number and integer number.
\end{enumerate}

\vspace{6mm}

\item {\em There are two variables {\tt a} and {\tt b} whose values need to be swapped. Which of the 
following codes will do it?}\\

\begin{enumerate}
\item[A1] 
\begin{verbatim}
a = b = a
\end{verbatim}
\item[A2] 
\begin{verbatim}
a = b
b = a
\end{verbatim}
\item[A3] 
\begin{verbatim}
a = b
c = a
b = a
\end{verbatim}
\item[A4] 
\begin{verbatim}
c = a
a = b
b = c
\end{verbatim}
\end{enumerate}

\vspace{6mm}

\item {\em Should we preferably use local variables, or global variables, and why?}\\

\begin{enumerate}
\item[A1] Global variables because they are easily accessible from any part of the code.
\item[A2] Global variables because they make our program faster.
\item[A3] Local variables because then we can use shadowing.
\item[A4] Local variables because our code is less prone to mistakes. 
\end{enumerate}

\vspace{6mm}

\item {\em Should we use global variables in functions and why?}\\

\begin{enumerate}
\item[A1] Yes, the function definition is simpler.
\item[A2] Yes, the program is less prone to mistakes.
\item[A3] No, it makes the code less transparent and more prone to mistakes.
\item[A4] No, Python does not allow it.
\end{enumerate}

\vspace{6mm}

\item {\em What is "shadowing of variables"?}\\

\begin{enumerate}
\item[A1] There are two or more functions that all use a local variable of the same name.
\item[A2] The type of a global variable is changed by assigning value of a different type to it. 
\item[A3] The type of a local variable is changed by assigning value of a different type to it. 
\item[A4] There is a local variable whose name matches the name of a global one.
\end{enumerate}

\vspace{6mm}

\item {\em What will be the output of the following code?}\\

\begin{Verbatim}[commandchars=\\\{\}]
\PY{n}{val} \PY{o}{=} \PY{l+m+mf}{5.0}
\PY{k}{def} \PY{n+nf}{power}\PY{p}{(}\PY{n}{x}\PY{p}{,} \PY{n}{p}\PY{p}{)}\PY{p}{:}
    \PY{n}{val} \PY{o}{=} \PY{n}{x}\PY{o}{*}\PY{o}{*}\PY{n}{p}
    \PY{k}{return} \PY{n}{val}
\PY{n}{result} \PY{o}{=} \PY{n}{power}\PY{p}{(}\PY{l+m+mi}{3}\PY{p}{,} \PY{l+m+mi}{2}\PY{p}{)}
\PY{k}{print} \PY{n}{val}
\end{Verbatim}
\vspace{6mm}

\begin{enumerate}
\item[A1] {\tt 9.0}
\item[A2] {\tt 5.0}
\item[A3] Error message
\item[A4] 
\begin{verbatim}
9.0
5.0
\end{verbatim}
\end{enumerate}

\vspace{6mm}

\item {\em Identify the output of the following code!}\\

\begin{Verbatim}[commandchars=\\\{\}]
\PY{k+kn}{from} \PY{n+nn}{numpy} \PY{k+kn}{import} \PY{n}{e}
\PY{k}{def} \PY{n+nf}{add}\PY{p}{(}\PY{n}{c}\PY{p}{,} \PY{n}{d}\PY{p}{)}\PY{p}{:}
    \PY{k}{return} c + d
e = \PY{n}{add}\PY{p}{(}\PY{l+m+mi}{10}\PY{p}{,} \PY{l+m+mi}{20}\PY{p}{)}
\PY{k}{print} \PY{n}{e}
\end{Verbatim}

\vspace{6mm}

\begin{enumerate}
\item[A1] {\tt 30}
\item[A2] {\tt 2.718281828459045}
\item[A3] {\tt 32.718281828459045}
\item[A4] Error message
\end{enumerate}

\vspace{6mm}

\end{enumerate}

%%%%%%%%%%%%%%%%%%%%%%%%%%%%%%%%%%%%%%%%%%%%%%%%%%%%%%%%%%%%%%%%%%%%%

\section{Logic and Probability}

\begin{enumerate}

\item {\em What is the output of the following code?}\\

\begin{Verbatim}[commandchars=\\\{\}]
\PY{n}{a} \PY{o}{=} \PY{n+nb+bp}{True}
\PY{n}{b} \PY{o}{=} \PY{n+nb+bp}{False}
\PY{n}{c} \PY{o}{=} \PY{n}{a} \PY{o+ow}{and} \PY{n}{b}
\PY{k}{print} \PY{n}{c}
\end{Verbatim}
\vspace{6mm}

\begin{enumerate}
\item[A1] {\tt False}
\item[A2] Undefined
\item[A3] {\tt True}
\item[A4] Error message
\end{enumerate}

\vspace{6mm}

\item {\em What is the output of this code?}\\

\begin{Verbatim}[commandchars=\\\{\}]
\PY{n}{a} \PY{o}{=} \PY{n+nb+bp}{True}
\PY{n}{b} \PY{o}{=} \PY{n+nb+bp}{False}
\PY{n}{c} \PY{o}{=} \PY{n}{a} \PY{o+ow}{or} \PY{n}{b}
\PY{k}{print} \PY{n}{c}
\end{Verbatim}
\vspace{6mm}

\begin{enumerate}
\item[A1] {\tt True}
\item[A2] Error message
\item[A3] {\tt False}
\item[A4] Undefined
\end{enumerate}

\vspace{6mm}

\item {\em What value will the variable {\tt d} have after the following code is run?}\\

\begin{Verbatim}[commandchars=\\\{\}]
\PY{n}{d} \PY{o}{=} \PY{n+nb+bp}{True}
\PY{k}{if} \PY{n}{d} \PY{o}{!=} \PY{n+nb+bp}{False}\PY{p}{:}
    \PY{n}{d} \PY{o}{=} \PY{o+ow}{not}\PY{p}{(}\PY{n}{d}\PY{p}{)}
\end{Verbatim}
\vspace{6mm}

\begin{enumerate}
\item[A1] {\tt True}
\item[A2] {\tt False}
\item[A3] {\tt True} in Python 2.7 and {\tt False} in Python 3.0
\item[A4] Undefined
\end{enumerate}

\vspace{6mm}

\item {\em What will the following code print?}\\

\begin{Verbatim}[commandchars=\\\{\}]
\PY{n}{val} \PY{o}{=} \PY{l+m+mi}{2} \PY{o}{+} \PY{l+m+mi}{3}
\PY{k}{if} \PY{n}{val} \PY{o}{=} \PY{l+m+mi}{6}\PY{p}{:}
    \PY{k}{print} \PY{l+s}{"}\PY{l+s}{Correct result.}\PY{l+s}{"}
\PY{k}{else}\PY{p}{:}
    \PY{k}{print} \PY{l+s}{"}\PY{l+s}{Wrong result.}\PY{l+s}{"}
\end{Verbatim}
\vspace{6mm}

\begin{enumerate}
\item[A1] {\tt Correct result.}
\item[A2] {\tt Wrong result.}
\item[A3] {\tt 5}
\item[A4] Error message
\end{enumerate}

\vspace{6mm}

\item {\em What will be the value of the variable {\tt var} after the following code is run?}\\

\begin{Verbatim}[commandchars=\\\{\}]
\PY{n}{n} \PY{o}{=} \PY{l+m+mi}{10} \PY{o}{+} \PY{l+m+mi}{5}
\PY{n}{m} \PY{o}{=} \PY{l+m+mi}{4} \PY{o}{*} \PY{l+m+mi}{5}
\PY{n}{var} \PY{o}{=} \PY{n}{n} \PY{o}{==} \PY{n}{m}
\end{Verbatim}
\vspace{6mm}

\begin{enumerate}
\item[A1] {\tt 15}
\item[A2] {\tt 20}
\item[A3] {\tt False}
\item[A4] {\tt True}
\end{enumerate}

\vspace{6mm}

\item {\em Let {\tt a} and {\tt b} be Boolean variables. What is the 
output of the following code?}\\

\begin{Verbatim}[commandchars=\\\{\}]
\PY{n}{c} \PY{o}{=} \PY{p}{(}\PY{n}{a} \PY{o+ow}{or} \PY{n}{b}\PY{p}{)} \PY{o+ow}{or} \PY{o+ow}{not} \PY{p}{(}\PY{n}{a} \PY{o+ow}{or} \PY{n}{b}\PY{p}{)}
\PY{k}{print} \PY{n}{c}
\end{Verbatim}
\vspace{6mm}

\begin{enumerate}
\item[A1] {\tt False}
\item[A2] Undefined.
\item[A3] {\tt False} or {\tt True}, depending on the values of {\tt a} and {\tt b}.
\item[A4] {\tt True}
\end{enumerate}

\vspace{6mm}

\item {\em What are Monte Carlo methods in scientific computing?}\\

\begin{enumerate}
\item[A1] Methods that use large numbers of random values.
\item[A2] Methods that succeed or fail randomly.
\item[A3] Methods whose outcome is either {\tt True} or {\tt False}
\item[A4] Methods for efficient evaluation of Boolean expressions..
\end{enumerate}

\vspace{6mm}

\item {\em Which of the four values below will be closest to the output of this program?}\\

\begin{Verbatim}[commandchars=\\\{\}]
\PY{k+kn}{from} \PY{n+nn}{random} \PY{k+kn}{import} \PY{n}{random}
\PY{n}{n} \PY{o}{=} \PY{l+m+mi}{1000000}
\PY{n}{m} \PY{o}{=} \PY{l+m+mi}{0}
\PY{c}{\PYZsh{} Repeat n times:}
\PY{k}{for} \PY{n}{i} \PY{o+ow}{in} \PY{n+nb}{range}\PY{p}{(}\PY{n}{n}\PY{p}{)}\PY{p}{:}
    \PY{c}{\PYZsh{} Generate random real number between -2 and 2:}
    \PY{n}{x} \PY{o}{=} \PY{l+m+mi}{4} \PY{o}{*} \PY{n}{random}\PY{p}{(}\PY{p}{)} \PY{o}{-} \PY{l+m+mi}{2}
    \PY{k}{if} \PY{n}{x}\PY{o}{*}\PY{o}{*}\PY{l+m+mi}{2} \PY{o}{-} \PY{l+m+mi}{1} \PY{o}{<} \PY{l+m+mi}{0}\PY{p}{:}
        \PY{n}{m} \PY{o}{+}\PY{o}{=} \PY{l+m+mi}{1}
\PY{c}{\PYZsh{} Print the ratio of m and n:}
\PY{k}{print} \PY{n+nb}{float}\PY{p}{(}\PY{n}{m}\PY{p}{)} \PY{o}{/} \PY{n}{n} 
\end{Verbatim}
\vspace{6mm}

\begin{enumerate}
\item[A1] {\tt 0.0}
\item[A2] {\tt 4.0}
\item[A3] {\tt 0.5}
\item[A4] {\tt 1.0}
\end{enumerate}

\vspace{6mm}

\end{enumerate}

%%%%%%%%%%%%%%%%%%%%%%%%%%%%%%%%%%%%%%%%%%%%%%%%%%%%%%%%%%%%%%%%%%%%%

\section{Conditional Loop}

\begin{enumerate}

\item {\em When should the {\tt elif} statement be used?}\\

\begin{enumerate}
\item[A1] It should not be used, it is a bad programming practice.
\item[A2] When there are more than two cases in the {\tt if - else} statement. 
\item[A3] When the logical expression in the {\tt if} branch is not likely to be True.
\item[A4] When the logical expression in the {\tt if} branch is likely to be True.
\end{enumerate}

\vspace{6mm}

\item {\em What will be the output of the following program?}\\

\begin{Verbatim}[commandchars=\\\{\}]
\PY{n}{day} \PY{o}{=} \PY{l+m+mi}{11}
\PY{k}{if} \PY{n}{day} \PY{o}{==} \PY{l+m+mi}{1}\PY{p}{:}
    \PY{k}{print} \PY{l+s}{"}\PY{l+s}{Monday}\PY{l+s}{"}
\PY{k}{elif} \PY{n}{day} \PY{o}{==} \PY{l+m+mi}{2}\PY{p}{:}
    \PY{k}{print} \PY{l+s}{"}\PY{l+s}{Tuesday}\PY{l+s}{"}
\PY{k}{elif} \PY{n}{day} \PY{o}{==} \PY{l+m+mi}{3}\PY{p}{:}
    \PY{k}{print} \PY{l+s}{"}\PY{l+s}{Wednesday}\PY{l+s}{"}
\PY{k}{elif} \PY{n}{day} \PY{o}{==} \PY{l+m+mi}{4}\PY{p}{:}
    \PY{k}{print} \PY{l+s}{"}\PY{l+s}{Thursday}\PY{l+s}{"}
\PY{k}{elif} \PY{n}{day} \PY{o}{==} \PY{l+m+mi}{5}\PY{p}{:}
    \PY{k}{print} \PY{l+s}{"}\PY{l+s}{Friday}\PY{l+s}{"}
\PY{k}{elif} \PY{n}{day} \PY{o}{==} \PY{l+m+mi}{6}\PY{p}{:}
    \PY{k}{print} \PY{l+s}{"}\PY{l+s}{Saturday}\PY{l+s}{"}
\PY{k}{else}\PY{p}{:}
    \PY{k}{print} \PY{l+s}{"}\PY{l+s}{Sunday}\PY{l+s}{"}
\end{Verbatim}
\vspace{6mm}


\begin{enumerate}
\item[A1] There is no output.
\item[A2] 
\begin{verbatim}
Monday
Tuesday
Wednesday
Thursday
Friday
Saturday
Sunday
\end{verbatim}
\item[A3] 
\begin{verbatim}
Sunday
\end{verbatim}
\item[A4] Error message
\begin{verbatim}
\end{verbatim}
\end{enumerate}

\vspace{6mm}

\item {\em When should the {\tt while} loop be used?}\\

\begin{enumerate}
\item[A1] When we cannot use the {\tt if - else} condition.
\item[A2] When we know exactly how many repetitions will be done.
\item[A3] When the loop contains a variable that decreases to zero.
\item[A4] When the number of repetitions is not know a priori.
\end{enumerate}

\vspace{6mm}

\item {\em What will be the output of the following program?}\\

\begin{Verbatim}[commandchars=\\\{\}]
\PY{n}{n} \PY{o}{=} \PY{l+m+mi}{1}
\PY{k}{while} \PY{n}{n} \PY{o}{<} \PY{l+m+mi}{100}\PY{p}{:}
    \PY{n}{n} \PY{o}{*}\PY{o}{=} \PY{l+m+mi}{2}
\PY{k}{print} \PY{n}{n}
\end{Verbatim}
\vspace{6mm}

\begin{enumerate}
\item[A1] {\tt 64}
\item[A2] {\tt 128}
\item[A3] {\tt 1}
\item[A4] This is an infinite loop, there is no output.
\end{enumerate}

\vspace{6mm}

\item {\em What is the purpose of the {\tt break} statement?}\\

\begin{enumerate}
\item[A1] Stop the program inside of a loop.
\item[A2] Exit a loop. If multiple loops are embedded, exit all of them.
\item[A3] Exit the body of an {\tt if} or {\tt else} statement.
\item[A4] Exit a loop. If multiple loops are embedded, exit just the closest one.
\end{enumerate}

\vspace{6mm}

\item {\em What is the purpose of the {\tt continue} statement?}\\

\begin{enumerate}
\item[A1] Continue repeating the body of the loop after the loop has finished.
\item[A2] Skip the rest of the loop's body and continue with next cycle.
\item[A3] Continue to the next command.
\item[A4] Continue to the first line after the loop's body.
\end{enumerate}

\vspace{6mm}

\item {\em What will be the output of the following program?}\\

\begin{Verbatim}[commandchars=\\\{\}]
\PY{n}{a} \PY{o}{=} \PY{l+m+mi}{0}
\PY{k}{while} \PY{n+nb+bp}{True}\PY{p}{:}
    \PY{n}{a} \PY{o}{+}\PY{o}{=} \PY{l+m+mi}{1}
    \PY{k}{if} \PY{n}{a} \PY{o}{<} \PY{l+m+mi}{8}\PY{p}{:}
        \PY{k}{continue}
    \PY{k}{print} \PY{n}{a}
    \PY{k}{break}
\end{Verbatim}
\begin{enumerate}
\item[A1] 
\begin{verbatim}
0
1
2
3
4
5
6
7
8
\end{verbatim}
\item[A2] 
\begin{verbatim}
1
2
3
4
5
6
7
8
\end{verbatim}
\item[A3] 
\begin{verbatim}
1
2
3
4
5
6
7
\end{verbatim}
\item[A4] 
\begin{verbatim}
8
\end{verbatim}
\end{enumerate}

\vspace{6mm}

\item {\em What is the Newton's method in scientific computing?}\\

\begin{enumerate}
\item[A1] Method to determine force from mass and gravity.
\item[A2] Method to approximate solutions to nonlinear equations.
\item[A3] Method to calculate Newton integrals.
\item[A4] Method to determine duration of free fall of an apple.
\end{enumerate}

\vspace{6mm}

\end{enumerate}

%%%%%%%%%%%%%%%%%%%%%%%%%%%%%%%%%%%%%%%%%%%%%%%%%%%%%%%%%%%%%%%%%%%%%

\section{Strings}

\begin{enumerate}
\item {\em What of the following are correct ways to include quotes in a string?}\\

\begin{enumerate}
\item[A1] 
\begin{verbatim}
"I say "goodbye", you say "hello""
\end{verbatim}
\item[A2] 
\begin{verbatim}
"I say /"goodbye/", you say /"hello/""
\end{verbatim}
\item[A3] 
\begin{verbatim}
"I say \"goodbye\", you say \"hello\""
\end{verbatim}
\item[A4] 
\begin{verbatim}
"I say \'goodbye\', you say \'hello\'"
\end{verbatim}
\end{enumerate}

\vspace{6mm}

\item {\em What of the following are correct ways to define a multiline string?}\\

\begin{enumerate}
\item[A1] 
\begin{verbatim}
/*
I say "High", you say "Low".
You say "Why?" And I say "I don't know".
*/
\end{verbatim}
\item[A2] 
\begin{verbatim}
"""\
I say "High", you say "Low".
You say "Why?" And I say "I don't know".\
"""
\end{verbatim}
\item[A3] 
\begin{verbatim}
I say "High", you say "Low". \
You say "Why?" And I say "I don't know".
\end{verbatim}
\item[A4] 
\begin{verbatim}
"I say "High", you say "Low". \
You say "Why?" And I say "I don't know"."
\end{verbatim}
\end{enumerate}

\vspace{6mm}
\item {\em What output will be produced by the following code?}\\

\begin{Verbatim}[commandchars=\\\{\}]
\PY{n}{s1} \PY{o}{=} \PY{l+s}{"}\PY{l+s}{Thank you}\PY{l+s}{"}
\PY{n}{s2} \PY{o}{=} \PY{l+s}{"}\PY{l+s}{very}\PY{l+s}{"}
\PY{n}{s3} \PY{o}{=} \PY{l+s}{"}\PY{l+s}{much!}\PY{l+s}{"}
\PY{k}{print} \PY{n}{s1} \PY{o}{+} \PY{l+m+mi}{5}\PY{o}{*}\PY{n}{s2} \PY{o}{+} \PY{n}{s3}
\end{Verbatim}
\vspace{6mm}

\begin{enumerate}
\item[A1] 
\begin{verbatim}
Thank you very very very very very much!
\end{verbatim}
\item[A2] 
\begin{verbatim}
Thank youveryveryveryveryverymuch!
\end{verbatim}
\item[A3] 
\begin{verbatim}
Thankyouveryveryveryveryverymuch!
\end{verbatim}
\item[A4] None - an error will be thrown.
\end{enumerate}

\vspace{6mm}

\item {\em What output will be produced by the following code?}\\

\begin{Verbatim}[commandchars=\\\{\}]
\PY{n}{s1} \PY{o}{=} \PY{l+s}{"}\PY{l+s}{intermediate}\PY{l+s}{"}
\PY{n}{s2} \PY{o}{=} \PY{n}{s1}\PY{p}{[}\PY{l+m+mi}{7}\PY{p}{]} \PY{o}{+} \PY{n}{s1}\PY{p}{[}\PY{l+m+mi}{4}\PY{p}{]} \PY{o}{+} \PY{n}{s1}\PY{p}{[}\PY{l+m+mi}{3}\PY{p}{]} \PY{o}{+} \PY{n}{s1}\PY{p}{[}\PY{l+m+mi}{6}\PY{p}{]} \PY{o}{+} \PY{n}{s1}\PY{p}{[}\PY{l+m+mi}{5}\PY{p}{]}
\PY{k}{print} \PY{n}{s2}
\end{Verbatim}
\vspace{6mm}

\begin{enumerate}
\item[A1] 
\begin{verbatim}
dreem
\end{verbatim}
\item[A2] 
\begin{verbatim}
dream
\end{verbatim}
\item[A3] 
\begin{verbatim}
dieta
\end{verbatim}
\item[A4] 
\begin{verbatim}
tamer
\end{verbatim}
\end{enumerate}

\vspace{6mm}

\item {\em What output will be produced by the following code?}\\

\begin{Verbatim}[commandchars=\\\{\}]
\PY{n}{s1} \PY{o}{=} \PY{l+s}{"}\PY{l+s}{intermediate}\PY{l+s}{"}
\PY{n}{s2} \PY{o}{=} \PY{n}{s1}\PY{p}{[}\PY{p}{:}\PY{l+m+mi}{5}\PY{p}{]}
\PY{k}{print} \PY{n}{s2}
\PY{n}{s3} \PY{o}{=} \PY{n}{s1}\PY{p}{[}\PY{l+m+mi}{5}\PY{p}{:}\PY{l+m+mi}{10}\PY{p}{]}
\PY{k}{print} \PY{n}{s3}
\PY{n}{s4} \PY{o}{=} \PY{n}{s1}\PY{p}{[}\PY{o}{-}\PY{l+m+mi}{2}\PY{p}{]} \PY{o}{+} \PY{n}{s1}\PY{p}{
[}\PY{o}{-}\PY{l+m+mi}{1}\PY{p}{]}
\PY{k}{print} \PY{n}{s4}
\end{Verbatim}
\vspace{6mm}

\begin{enumerate}
\item[A1] 
\begin{verbatim}
intermediate
\end{verbatim}
\item[A2] 
\begin{verbatim}
inter
media
et
\end{verbatim}
\item[A3] 
\begin{verbatim}
inter
media
te
\end{verbatim}
\item[A4] None - an error will be thrown.
\end{enumerate}

\vspace{6mm}

\item {\em What is the correct way to measure and print the length of a string {\tt str}?}\\

\begin{enumerate}
\item[A1] 
\begin{Verbatim}[commandchars=\\\{\}]
\PY{k}{print} \PY{n}{length}\PY{p}{(}\PY{n}{str}\PY{p}{)}
\end{Verbatim}
\item[A2] 
\begin{Verbatim}[commandchars=\\\{\}]
\PY{k}{print} \PY{n}{len}\PY{p}{(}\PY{n}{str}\PY{p}{)}
\end{Verbatim}
\item[A3] 
\begin{Verbatim}[commandchars=\\\{\}]
\PY{k}{print} \PY{n}{abs}\PY{p}{(}\PY{n}{str}\PY{p}{)}
\end{Verbatim}
\item[A4] 
\begin{Verbatim}[commandchars=\\\{\}]
\PY{k}{print} str[0]
\end{Verbatim}
\end{enumerate}

\vspace{6mm}

\item {\em What will be the output of the following code?}\\

\begin{Verbatim}[commandchars=\\\{\}]
\PY{k}{print} \PY{n+nb}{range}\PY{p}{(}\PY{l+m+mi}{2}\PY{p}{,} \PY{l+m+mi}{5}\PY{p}{)}
\end{Verbatim}
\vspace{6mm}

\begin{enumerate}
\item[A1] 
\begin{verbatim}
[2, 3, 4, 5]
\end{verbatim}
\item[A2] 
\begin{verbatim}
[2, 3, 4, 5, 6]
\end{verbatim}
\item[A3] 
\begin{verbatim}
[2, 3, 4]
\end{verbatim}
\item[A4] 
\begin{verbatim}
[5, 6]
\end{verbatim}
\end{enumerate}

\vspace{6mm}

\item {\em What output corresponds to the following code?}\\

\begin{Verbatim}[commandchars=\\\{\}]
\PY{n}{word} \PY{o}{=} \PY{l+s}{"}\PY{l+s}{breakfast}\PY{l+s}{"}
\PY{k}{for} \PY{n}{m} \PY{o+ow}{in} \PY{n+nb}{range}\PY{p}{(}\PY{l+m+mi}{5}\PY{p}{,} \PY{l+m+mi}{9}\PY{p}{)}\PY{p}{:}
    \PY{k}{print} \PY{n}{word}\PY{p}{[}\PY{n}{m}\PY{p}{]}
\end{Verbatim}
\vspace{6mm}

\begin{enumerate}
\item[A1] 
\begin{verbatim}
fast
\end{verbatim}
\item[A2] 
\begin{verbatim}
kfas
\end{verbatim}
\item[A3] 
\begin{verbatim}
f
a
s
t
\end{verbatim}
\item[A4] 
\begin{verbatim}
kfas
\end{verbatim}
\end{enumerate}

\vspace{6mm}

\end{enumerate}

%%%%%%%%%%%%%%%%%%%%%%%%%%%%%%%%%%%%%%%%%%%%%%%%%%%%%%%%%%%%%%%%%%%%%

\section{Tuples, Lists, and Dictionaries}

\begin{enumerate}
\item {\em What is the variable {\tt var}?}\\[-2mm]

\begin{verbatim}
var = (1, 2, 3, 'A', 'B', 'C', "alpha", "beta", "gamma")
\end{verbatim}
\vspace{4mm}

\begin{enumerate}
\item[A1] List.
\item[A2] Tuple.
\item[A3] Dictionary.
\item[A4] None of the above.
\end{enumerate}

\vspace{6mm}

\item {\em What will be the output of the following program?}\\

\begin{Verbatim}[commandchars=\\\{\}]
\PY{n}{var} \PY{o}{=} \PY{p}{(}\PY{l+m+mi}{1}\PY{p}{,} \PY{l+m+mi}{2}\PY{p}{,} \PY{l+m+mi}{3}\PY{p}{,} \PY{l+s}{'}\PY{l+s}{A}\PY{l+s}{'}\PY{p}{,} \PY{l+s}{'}\PY{l+s}{B}\PY{l+s}{'}\PY{p}{,} \PY{l+s}{'}\PY{l+s}{C}\PY{l+s}{'}\PY{p}{,} \PY{l+s}{"}\PY{l+s}{alpha}\PY{l+s}{"}\PY{p}{,} \PY{l+s}{"}\PY{l+s}{beta}\PY{l+s}{"}\PY{p}{,} \PY{l+s}{"}\PY{l+s}{gamma}\PY{l+s}{"}\PY{p}{)}
\PY{k}{print} \PY{n}{var}\PY{p}{[}\PY{l+m+mi}{5}\PY{p}{]}
\PY{k}{print} \PY{n}{var}\PY{p}{[}\PY{p}{:}\PY{l+m+mi}{3}\PY{p}{]}
\PY{k}{print} \PY{n}{var}\PY{p}{[}\PY{l+m+mi}{6}\PY{p}{:}\PY{l+m+mi}{8}\PY{p}{]}
\end{Verbatim}
\vspace{6mm}

\begin{enumerate}
\item[A1] 
\begin{verbatim}
B
(1, 2, 3)
('alpha', 'beta', 'gamma')
\end{verbatim}
\item[A2] 
\begin{verbatim}
C
(1, 2, 3)
('alpha', 'beta')
\end{verbatim}
\item[A3] 
\begin{verbatim}
B
(1, 2, 3)
('C', 'alpha', 'beta', 'gamma')
\end{verbatim}
\item[A4]
\begin{verbatim}
B
(1, 2, 3)
('alpha', 'beta')
\end{verbatim}
\end{enumerate}

\vspace{6mm}

\item {\em Can new items be added to a tuple?}\\

\begin{enumerate}
\item[A1] Yes but only to an empty tuple.
\item[A2] Yes but only if all items are of the same type.
\item[A3] No.
\item[A4] Yes but only if not all items are of the same type.
\end{enumerate}

\vspace{6mm}

\item {\em What is the correct way to determine the length of a tuple {\tt T}?}\\

\begin{enumerate}
\item[A1] 
\begin{verbatim}
length(T)
\end{verbatim}
\item[A2] 
\begin{verbatim}
tlength(T)
\end{verbatim}
\item[A3] 
\begin{verbatim}
len(T)
\end{verbatim}
\item[A4] 
\begin{verbatim}
tlen(T)
\end{verbatim}
\end{enumerate}

\vspace{6mm}

\item {\em What is the variable {\tt names}?}\\

\begin{Verbatim}[commandchars=\\\{\}]
\PY{n}{names} \PY{o}{=} \PY{p}{[}\PY{l+s}{"}\PY{l+s}{John}\PY{l+s}{"}\PY{p}{,} \PY{l+s}{"}\PY{l+s}{Jake}\PY{l+s}{"}\PY{p}{,} \PY{l+s}{"}\PY{l+s}{Josh}\PY{l+s}{"}\PY{p}{]}
\end{Verbatim}
\vspace{6mm}

\begin{enumerate}
\item[A1] List.
\item[A2] Tuple.
\item[A3] Dictionary.
\item[A4] None of the above.
\end{enumerate}

\vspace{6mm}

\item {\em Identify the output of the following code!}\\

\begin{Verbatim}[commandchars=\\\{\}]
\PY{n}{names} \PY{o}{=} \PY{p}{[}\PY{l+s}{"}\PY{l+s}{John}\PY{l+s}{"}\PY{p}{,} \PY{l+s}{"}\PY{l+s}{Jake}\PY{l+s}{"}\PY{p}{,} \PY{l+s}{"}\PY{l+s}{Josh}\PY{l+s}{"}\PY{p}{]}
\PY{n}{name} \PY{o}{=} \PY{n}{names}\PY{o}{.}\PY{k}{del}\PY{p}{[}\PY{l+m+mi}{1}\PY{p}{]}
\PY{k}{print} \PY{n}{name}
\end{Verbatim}
\vspace{6mm}

\begin{enumerate}
\item[A1] 
\begin{verbatim}
John
\end{verbatim}
\item[A2] 
\begin{verbatim}
Jake
\end{verbatim}
\item[A3] 
\begin{verbatim}
Josh
\end{verbatim}
\item[A4] Error message.
\end{enumerate}

\vspace{6mm}

\item {\em Identify the output of the following code!}\\

\begin{Verbatim}[commandchars=\\\{\}]
\PY{n}{names} \PY{o}{=} \PY{p}{[}\PY{l+s}{"}\PY{l+s}{John}\PY{l+s}{"}\PY{p}{,} \PY{l+s}{"}\PY{l+s}{Jake}\PY{l+s}{"}\PY{p}{,} \PY{l+s}{"}\PY{l+s}{Josh}\PY{l+s}{"}\PY{p}{]}
\PY{n}{names}\PY{o}{.}\PY{n}{append}\PY{p}{(}\PY{l+s}{"}\PY{l+s}{Jerry}\PY{l+s}{"}\PY{p}{)}
\PY{k}{print} \PY{n}{names}
\end{Verbatim}
\vspace{6mm}

\begin{enumerate}
\item[A1] 
\begin{verbatim}
['John', 'Jake', 'Josh', 'Jerry']
\end{verbatim}
\item[A2] 
\begin{verbatim}
('John', 'Jake', 'Josh', 'Jerry')
\end{verbatim}
\item[A3] 
\begin{verbatim}
['Jerry', 'John', 'Jake', 'Josh']
\end{verbatim}
\item[A4] Error message.
\end{enumerate}

\vspace{6mm}

\item {\em Identify the output of the following code!}\\

\begin{Verbatim}[commandchars=\\\{\}]
\PY{n}{names} \PY{o}{=} \PY{p}{[}\PY{l+s}{"}\PY{l+s}{John}\PY{l+s}{"}\PY{p}{,} \PY{l+s}{"}\PY{l+s}{Jake}\PY{l+s}{"}\PY{p}{,} \PY{l+s}{"}\PY{l+s}{Josh}\PY{l+s}{"}\PY{p}{]}
\PY{n}{names}\PY{o}{.}\PY{n}{pop}\PY{p}{(}\PY{l+m+mi}{0}\PY{p}{)}
\PY{k}{print} \PY{n}{names}
\end{Verbatim}
\vspace{6mm}

\begin{enumerate}
\item[A1] 
\begin{verbatim}
['Jake', 'Josh']
\end{verbatim}
\item[A2] 
\begin{verbatim}
('John', 'Jake', 'Josh')
\end{verbatim}
\item[A3] 
\begin{verbatim}
John
\end{verbatim}
\item[A4] Error message.
\end{enumerate}

\vspace{6mm}

\item {\em Identify the output of the following code!}\\

\begin{Verbatim}[commandchars=\\\{\}]
\PY{n}{names} \PY{o}{=} \PY{p}{[}\PY{l+s}{"}\PY{l+s}{John}\PY{l+s}{"}\PY{p}{,} \PY{l+s}{"}\PY{l+s}{Jake}\PY{l+s}{"}\PY{p}{,} \PY{l+s}{"}\PY{l+s}{Josh}\PY{l+s}{"}\PY{p}{]}
\PY{n}{names}\PY{o}{.}\PY{n}{insert}\PY{p}{(}\PY{l+m+mi}{1}\PY{p}{,} \PY{l+s}{"}\PY{l+s}{Jenny}\PY{l+s}{"}\PY{p}{)}
\PY{k}{print} \PY{n}{names}
\end{Verbatim}
\vspace{6mm}

\begin{enumerate}
\item[A1] 
\begin{verbatim}
['Jenny', 'John', 'Jake', 'Josh']
\end{verbatim}
\item[A2] 
\begin{verbatim}
['John', 'Jenny', 'Jake', 'Josh']
\end{verbatim}
\item[A3] 
\begin{verbatim}
['Jenny', 'Jake', 'Josh']
\end{verbatim}
\item[A4] Error message.
\end{enumerate}

\vspace{6mm}

\item {\em Identify the output of the following code!}\\

\begin{Verbatim}[commandchars=\\\{\}]
\PY{n}{names} \PY{o}{=} \PY{p}{[}\PY{l+s}{"}\PY{l+s}{John}\PY{l+s}{"}\PY{p}{,} \PY{l+s}{"}\PY{l+s}{Jake}\PY{l+s}{"}\PY{p}{,} \PY{l+s}{"}\PY{l+s}{Josh}\PY{l+s}{"}\PY{p}{]}
\PY{n}{names}\PY{o}{.}\PY{n}{reverse}\PY{p}{(}\PY{p}{)}
\PY{k}{print} \PY{n}{names}
\end{Verbatim}
\vspace{6mm}

\begin{enumerate}
\item[A1] 
\begin{verbatim}
['John', 'Josh', 'Jake']
\end{verbatim}
\item[A2] 
\begin{verbatim}
['Josh', 'John', 'Jake']
\end{verbatim}
\item[A3] 
\begin{verbatim}
['Josh', 'Jake', 'John']
\end{verbatim}
\item[A4] Error message.
\end{enumerate}

\vspace{6mm}

\item {\em Identify the output of the following code!}\\

\begin{Verbatim}[commandchars=\\\{\}]
\PY{n}{names} \PY{o}{=} \PY{p}{[}\PY{l+s}{"}\PY{l+s}{John}\PY{l+s}{"}\PY{p}{,} \PY{l+s}{"}\PY{l+s}{Jerry}\PY{l+s}{"}\PY{p}{,} \PY{l+s}{"}\PY{l+s}{Jake}\PY{l+s}{"}\PY{p}{,} \PY{l+s}{"}\PY{l+s}{Josh}\PY{l+s}{"}\PY{p}{,} \PY{l+s}{"}\PY{l+s}{Jerry}\PY{l+s}{"}\PY{p}{]}
\PY{n}{names}\PY{o}{.}\PY{n}{sort}\PY{p}{(}\PY{p}{)}
\PY{k}{print} \PY{n}{names}
\end{Verbatim}
\vspace{6mm}

\begin{enumerate}
\item[A1] 
\begin{verbatim}
['Jake', 'Jerry', 'John', 'Josh']
\end{verbatim}
\item[A2] 
\begin{verbatim}
['Jake', 'Jerry', 'Jerry', 'John', 'Josh']
\end{verbatim}
\item[A3] 
\begin{verbatim}
['Josh', 'John', 'Jerry', 'Jerry', 'Jake']
\end{verbatim}
\item[A4] Error message.
\end{enumerate}

\vspace{6mm}

\item {\em Identify the output of the following code!}\\

\begin{Verbatim}[commandchars=\\\{\}]
\PY{n}{names} \PY{o}{=} \PY{p}{[}\PY{l+s}{"}\PY{l+s}{John}\PY{l+s}{"}\PY{p}{,} \PY{l+s}{"}\PY{l+s}{Jerry}\PY{l+s}{"}\PY{p}{,} \PY{l+s}{"}\PY{l+s}{Jake}\PY{l+s}{"}\PY{p}{,} \PY{l+s}{"}\PY{l+s}{Josh}\PY{l+s}{"}\PY{p}{,} \PY{l+s}{"}\PY{l+s}{Jerry}\PY{l+s}{"}\PY{p}{]}
\PY{k}{print} \PY{n}{names}\PY{o}{.}\PY{n}{count}\PY{p}{(}\PY{l+s}{"}\PY{l+s}{Jerry}\PY{l+s}{"}\PY{p}{)}
\end{Verbatim}
\vspace{6mm}

\begin{enumerate}
\item[A1] 
\begin{verbatim}
1
\end{verbatim}
\item[A2] 
\begin{verbatim}
2
\end{verbatim}
\item[A3] 
\begin{verbatim}
(1, 4)
\end{verbatim}
\item[A4] 
\begin{verbatim}
(2, 5)
\end{verbatim}
\end{enumerate}

\vspace{6mm}

\item {\em Identify the output of the following code!}\\

\begin{Verbatim}[commandchars=\\\{\}]
\PY{n}{names} \PY{o}{=} \PY{p}{[}\PY{l+s}{"}\PY{l+s}{John}\PY{l+s}{"}\PY{p}{,} \PY{l+s}{"}\PY{l+s}{Jerry}\PY{l+s}{"}\PY{p}{,} \PY{l+s}{"}\PY{l+s}{Jake}\PY{l+s}{"}\PY{p}{,} \PY{l+s}{"}\PY{l+s}{Josh}\PY{l+s}{"}\PY{p}{,} \PY{l+s}{"}\PY{l+s}{Jerry}\PY{l+s}{"}\PY{p}{]}
\PY{k}{print} \PY{n}{names}\PY{o}{.}\PY{n}{index}\PY{p}{(}\PY{l+s}{"}\PY{l+s}{Jerry}\PY{l+s}{"}\PY{p}{)}
\end{Verbatim}
\vspace{6mm}

\begin{enumerate}
\item[A1] 
\begin{verbatim}
1
\end{verbatim}
\item[A2] 
\begin{verbatim}
(1, -1)
\end{verbatim}
\item[A3] 
\begin{verbatim}
(1, 4)
\end{verbatim}
\item[A4] 
\begin{verbatim}
(2, 5)
\end{verbatim}
\end{enumerate}

\vspace{6mm}

\item {\em Identify the output of the following code!}\\

\begin{Verbatim}[commandchars=\\\{\}]
\PY{n}{A} \PY{o}{=} \PY{p}{[}\PY{l+s}{"}\PY{l+s}{John}\PY{l+s}{"}\PY{p}{,} \PY{l+s}{"}\PY{l+s}{Jerry}\PY{l+s}{"}\PY{p}{,} \PY{l+s}{"}\PY{l+s}{Jed}\PY{l+s}{"}\PY{p}{]}
\PY{n}{B} \PY{o}{=} \PY{p}{[}\PY{l+s}{'}\PY{l+s}{1}\PY{l+s}{'}\PY{p}{,} \PY{l+s}{'}\PY{l+s}{2}\PY{l+s}{'}\PY{p}{,} \PY{l+s}{'}\PY{l+s}{3}\PY{l+s}{'}\PY{p}{]}
\PY{k}{print} \PY{n+nb}{zip}\PY{p}{(}\PY{n}{A}\PY{p}{,} \PY{n}{B}\PY{p}{)}
\end{Verbatim}
\vspace{6mm}

\begin{enumerate}
\item[A1] 
\begin{verbatim}
['John', 'Jerry', 'Jed', '1', '2', '3']
\end{verbatim}
\item[A2] 
\begin{verbatim}
[('John', 'Jerry', 'Jed'), ('1', '2', '3')]
\end{verbatim}
\item[A3] 
\begin{verbatim}
[('1', 'John'), ('2', 'Jerry'), ('3', 'Jed')]
\end{verbatim}
\item[A4] 
\begin{verbatim}
[('John', '1'), ('Jerry', '2'), ('Jed', '3')]
\end{verbatim}
\end{enumerate}

\vspace{6mm}

\item {\em What is the correct way to define a dictionary {\tt D} containing the 
      English words "city", "fire", "sun" and their Spanish translations "ciudad",
      "fuego", and "sol" ?}\\

\begin{enumerate}
\item[A1] 
\begin{verbatim}
D = {'city': 'ciudad', 'fire': 'fuego', 'sun': 'sol'}
\end{verbatim}
\item[A2] 
\begin{verbatim}
D = {'fire': 'fuego', 'sun': 'sol', 'city': 'ciudad'}
\end{verbatim}
\item[A3] 
\begin{verbatim}
D = {'sun': 'sol', 'fire': 'fuego', 'city': 'ciudad'}
\end{verbatim}
\item[A4] 
\begin{verbatim}
D = {'fire': 'fuego', 'city': 'ciudad', 'sun': 'sol'}
\end{verbatim}
\end{enumerate}

\vspace{6mm}

\item {\em What is the correct way to add to a dictionary {\tt D} new key 
"school" whose value is "escuela"?}\\

\begin{enumerate}
\item[A1] 
\begin{verbatim}
D.append('school': 'escuela')
\end{verbatim}
\item[A2] 
\begin{verbatim}
D.append_key('school')
D.append_value('escuela')
\end{verbatim}
\item[A3] 
\begin{verbatim}
D['school'] = 'escuela'
\end{verbatim}
\item[A4] 
\begin{verbatim}
D.add('school': 'escuela')
\end{verbatim}
\end{enumerate}

\vspace{6mm}

\item {\em What is the correct way to print the value for 
the key "city" in the dictionary {\tt D}?}\\

\begin{Verbatim}[commandchars=\\\{\}]
\PY{n}{D} \PY{o}{=} \PY{p}{\PYZob{}}\PY{l+s}{'}\PY{l+s}{fire}\PY{l+s}{'}\PY{p}{:} \PY{l+s}{'}\PY{l+s}{fuego}\PY{l+s}{'}\PY{p}{,} \PY{l+s}{'}\PY{l+s}{city}\PY{l+s}{'}\PY{p}{:} \PY{l+s}{'}\PY{l+s}{ciudad}\PY{l+s}{'}\PY{p}{,} \PY{l+s}{'}\PY{l+s}{sun}\PY{l+s}{'}\PY{p}{:} \PY{l+s}{'}\PY{l+s}{sol}\PY{l+s}{'}\PY{p}{\PYZcb{}}
\end{Verbatim}
\vspace{6mm}

\begin{enumerate}
\item[A1] 
\begin{verbatim}
D.print('city')
\end{verbatim}
\item[A2] 
\begin{verbatim}
print D['city']
\end{verbatim}
\item[A3] 
\begin{verbatim}
print D('city')
\end{verbatim}
\item[A4] 
\begin{verbatim}
print D.get_key('city')
\end{verbatim}
\end{enumerate}

\vspace{6mm}

\item {\em What is the correct way to ascertain whether
or not the key "sun" is present in the dictionary {\tt D}?}\\

\begin{Verbatim}[commandchars=\\\{\}]
\PY{n}{D} \PY{o}{=} \PY{p}{\PYZob{}}\PY{l+s}{'}\PY{l+s}{fire}\PY{l+s}{'}\PY{p}{:} \PY{l+s}{'}\PY{l+s}{fuego}\PY{l+s}{'}\PY{p}{,} \PY{l+s}{'}\PY{l+s}{city}\PY{l+s}{'}\PY{p}{:} \PY{l+s}{'}\PY{l+s}{ciudad}\PY{l+s}{'}\PY{p}{,} \PY{l+s}{'}\PY{l+s}{sun}\PY{l+s}{'}\PY{p}{:} \PY{l+s}{'}\PY{l+s}{sol}\PY{l+s}{'}\PY{p}{\PYZcb{}}
\end{Verbatim}
\vspace{6mm}

\begin{enumerate}
\item[A1] 
\begin{verbatim}
D.contains('sun')
\end{verbatim}
\item[A2] 
\begin{verbatim}
D.key('sun')
\end{verbatim}
\item[A3] 
\begin{verbatim}
D.try('sun')
\end{verbatim}
\item[A4] 
\begin{verbatim}
D.has_key('sun')
\end{verbatim}
\end{enumerate}

\vspace{6mm}

\item {\em What is the correct way to print 
all keys present in the dictionary {\tt D}?}\\

\begin{Verbatim}[commandchars=\\\{\}]
\PY{n}{D} \PY{o}{=} \PY{p}{\PYZob{}}\PY{l+s}{'}\PY{l+s}{fire}\PY{l+s}{'}\PY{p}{:} \PY{l+s}{'}\PY{l+s}{fuego}\PY{l+s}{'}\PY{p}{,} \PY{l+s}{'}\PY{l+s}{city}\PY{l+s}{'}\PY{p}{:} \PY{l+s}{'}\PY{l+s}{ciudad}\PY{l+s}{'}\PY{p}{,} \PY{l+s}{'}\PY{l+s}{sun}\PY{l+s}{'}\PY{p}{:} \PY{l+s}{'}\PY{l+s}{sol}\PY{l+s}{'}\PY{p}{\PYZcb{}}
\end{Verbatim}
\vspace{6mm}

\begin{enumerate}
\item[A1] 
\begin{verbatim}
print D.get_keys()
\end{verbatim}
\item[A2] 
\begin{verbatim}
print D.get_all()
\end{verbatim}
\item[A3] 
\begin{verbatim}
print get_keys(D)
\end{verbatim}
\item[A4] 
\begin{verbatim}
print D.keys()
\end{verbatim}
\end{enumerate}

\vspace{6mm}

\item {\em What is the correct way to print 
all values present in the dictionary {\tt D}?}\\

\begin{Verbatim}[commandchars=\\\{\}]
\PY{n}{D} \PY{o}{=} \PY{p}{\PYZob{}}\PY{l+s}{'}\PY{l+s}{fire}\PY{l+s}{'}\PY{p}{:} \PY{l+s}{'}\PY{l+s}{fuego}\PY{l+s}{'}\PY{p}{,} \PY{l+s}{'}\PY{l+s}{city}\PY{l+s}{'}\PY{p}{:} \PY{l+s}{'}\PY{l+s}{ciudad}\PY{l+s}{'}\PY{p}{,} \PY{l+s}{'}\PY{l+s}{sun}\PY{l+s}{'}\PY{p}{:} \PY{l+s}{'}\PY{l+s}{sol}\PY{l+s}{'}\PY{p}{\PYZcb{}}
\end{Verbatim}
\vspace{6mm}

\begin{enumerate}
\item[A1] 
\begin{verbatim}
print D.get_values()
\end{verbatim}
\item[A2] 
\begin{verbatim}
print D.get_all()
\end{verbatim}
\item[A3] 
\begin{verbatim}
print get_values(D)
\end{verbatim}
\item[A4] 
\begin{verbatim}
print D.values()
\end{verbatim}
\end{enumerate}

\vspace{6mm}

\end{enumerate}

%%%%%%%%%%%%%%%%%%%%%%%%%%%%%%%%%%%%%%%%%%%%%%%%%%%%%%%%%%%%%%%%%%%%%

\section{More on Counting Loop} 

\begin{enumerate}

\item {\em What language element in Karel the Robot is most similar to the {\tt for} loop in Python?}\\

\begin{enumerate}
\item[A1] The {\tt while} loop.
\item[A2] The {\tt if - else} statement.
\item[A3] The {\tt repeat} loop.
\item[A4] Recursion.
\end{enumerate}

\vspace{6mm}

\item {\em When should the {\tt for} loop be used?}\\

\begin{enumerate}
\item[A1] When we need to go through a list or tuple.
\item[A2] When the number of repetitions is known in advance. 
\item[A3] When the number of repetitions is not known in advance. 
\item[A4] To replace an infinite {\tt while} loop.
\end{enumerate}

\vspace{6mm}

\item {\em Which of the following codes will print numbers 4, 5, 6, 7, 8, 9?}\\

\begin{enumerate}
\item[A1] 
\begin{Verbatim}[commandchars=\\\{\}]
\PY{k}{for} \PY{n}{i} \PY{o+ow}{in} \PY{n+nb}{range}\PY{p}{(}\PY{l+m+mi}{4}\PY{p}{:}\PY{l+m+mi}{9}\PY{p}{)}\PY{p}{:}
    \PY{k}{print} \PY{n}{i}
\end{Verbatim}
\item[A2] 
\begin{Verbatim}[commandchars=\\\{\}]
\PY{n}{L} \PY{o}{=} \PY{n+nb}{range}\PY{p}{(}\PY{l+m+mi}{4}\PY{p}{,} \PY{l+m+mi}{10}\PY{p}{)}
\PY{k}{for} \PY{n}{number} \PY{o+ow}{in} \PY{n}{L}\PY{p}{:}
    \PY{k}{print} \PY{n}{number}
\end{Verbatim}
\item[A3] 
\begin{Verbatim}[commandchars=\\\{\}]
\PY{k}{for} \PY{n}{val} \PY{o+ow}{in} \PY{n+nb}{range}\PY{p}{(}\PY{l+m+mi}{4}\PY{p}{:}\PY{l+m+mi}{10}\PY{p}{)}\PY{p}{:}
    \PY{k}{print} \PY{n}{val}
\end{Verbatim}
\item[A4] 
\begin{Verbatim}[commandchars=\\\{\}]
\PY{k}{for} \PY{n}{i} \PY{o+ow}{in} \PY{n+nb}{range}\PY{p}{(}\PY{l+m+mi}{4}\PY{p}{:}\PY{l+m+mi}{10}\PY{p}{)}\PY{p}{:}
\PY{k}{print} \PY{n}{i}
\end{Verbatim}
\end{enumerate}

\vspace{6mm}

\end{enumerate}


%%%%%%%%%%%%%%%%%%%%%%%%%%%%%%%%%%%%%%%%%%%%%%%%%%%%%%%%%%%%%%%%%%%%%

\section{Exceptions}

\begin{enumerate}

\item {\em What is an {\em exception} in programming?}\\

\begin{enumerate}
\item[A1] Sequence of commands that is executed when an {\tt if} condition is not satisfied.
\item[A2] Exceptional situation in the code leading to a crash when not handled. 
\item[A3] Sequence of commands that is executed after an error is thrown.
\item[A4] Sequence of commands that is executed when an {\tt if} condition is not satisfied.
\end{enumerate}

\vspace{6mm}

\item {\em Which of the following are exceptions in Python.}\\

\begin{enumerate}
\item[A1] {\tt IndentationError}
\item[A2] {\tt UnboundLocalError}
\item[A3] {\tt TimeoutError}
\item[A4] {\tt OverflowError}
\end{enumerate}

\vspace{6mm}

\item {\em What does {\tt assert(x != 0)} do if {\tt x} is zero?}\\

\begin{enumerate}
\item[A1] Stops the program.
\item[A2] Raises the {\tt ZeroDivisionError} exception 
\item[A3] Raises the {\tt AssertionError} exception.
\item[A4] Prints a warning saying that {\tt x} is zero.
\end{enumerate}

\vspace{6mm}

\item {\em What is the correct way to raise a {\tt ValueError} exception 
      when {\tt x} is greater than five?}\\

\begin{enumerate}
\item[A1] 
\begin{verbatim}
if x > 5:
    ValueError("x should be <= five!")
\end{verbatim}
\item[A2] 
\begin{verbatim}
if x > 5:
    raise ValueError("x should be <= five!")
\end{verbatim}
\item[A3] 
\begin{verbatim}
if x > 5:
    exception ValueError("x should be <= five!")
\end{verbatim}
\item[A4] 
\begin{verbatim}
if x > 5:
    raise_exception ValueError("x should be <= five!")
\end{verbatim}
\end{enumerate}

\vspace{6mm}

\end{enumerate}

%%%%%%%%%%%%%%%%%%%%%%%%%%%%%%%%%%%%%%%%%%%%%%%%%%%%%%%%%%%%%%%%%%%%%

\section{Object-Oriented Programming}

\begin{enumerate}

\item {\em The philosophy of object-oriented programming is:}\\

\begin{enumerate}
\item[A1] Operate with geometrical objects.
\item[A2] Use local variables in functions.
\item[A3] Use entities that combine functionality and data.
\item[A4] Use functions that are local in other functions.
\end{enumerate}

\vspace{6mm}

\item {\em What is the relation between {\em class} and {\em object} in object-oriented programming?}\\

\begin{enumerate}
\item[A1] By an object we mean all the data defined in a class.
\item[A2] By an object we mean all the functions defined in a class.
\item[A3] Class is an instance (concrete realization) of an object.
\item[A4] Object is an instance (concrete realization) of a class.
\end{enumerate}

\vspace{6mm}

\item {\em What are {\em methods} of a class?}\\

\begin{enumerate}
\item[A1] Specific ways the class is defined.
\item[A2] Special functions defined in the class that operate with data not owned by the class.
\item[A3] Specific way variables are arranged in a class.
\item[A4] Functions that are part of the class definition.
\end{enumerate}

\vspace{6mm}

\item {\em Where are methods defined and where are they used?}\\

\begin{enumerate}
\item[A1] They are defined in an object and used by a class.
\item[A2] They are defined outside of a class and used inside of the class.
\item[A3] They are defined in a class and used by instances of the class.
\item[A4] They are defined in instances of a class.
\end{enumerate}

\vspace{6mm}

\item {\em Can methods of a class operate with data not owned by the class and when?}\\

\begin{enumerate}
\item[A1] Yes, always.
\item[A2] Yes, but only if they also operate with data owned by the class.
\item[A3] Yes, but only if they do not operate with data owned by the class.
\item[A4] No.
\end{enumerate}

\vspace{6mm}

\item {\em What is a {\em constructor}?}\\

\begin{enumerate}
\item[A1] Method of a class that is used to initialize newly created instances.
\item[A2] Special function that is defined in the object, not in the class.
\item[A3] Function that turns a class into an object.
\item[A4] Function that converts an object into a class.
\end{enumerate}

\vspace{6mm}

\item {\em What is the correct way to define a constructor that initializes a variable
      {\tt A} in a class with a value {\tt a}?}\\

\begin{enumerate}
\item[A1] 
\begin{verbatim}
    def __init__(a):
        A = a
\end{verbatim}
\item[A2] 
\begin{verbatim}
    def __init__(self, a):
        A = a
\end{verbatim}
\item[A3] 
\begin{verbatim}
    def __init__(self, a):
        self.A = a
\end{verbatim}
\item[A4] 
\begin{verbatim}
    def __init__(self, self.a):
        self.A = self.a
\end{verbatim}
\end{enumerate}

\vspace{6mm}

\item {\em A class contains a variable {\tt A}. What is the correct way to define 
      a method {\tt printdata} of this class that prints the value of the variable?}\\

\begin{enumerate}
\item[A1] 
\begin{verbatim}
    def printdata():
        print "A =", A
\end{verbatim}
\item[A2] 
\begin{verbatim}
    def printdata():
        print "A =", self.A
\end{verbatim}
\item[A3] 
\begin{verbatim}
    def printdata(self):
        print "A =", A
\end{verbatim}
\item[A4] 
\begin{verbatim}
    def printdata(self):
        print "A =", self.A
\end{verbatim}
\end{enumerate}

\vspace{6mm}

\item {\em Given a text string {\tt S}, what is the correct way to count and print the number 
      of appearances of another string {\tt word} in {\tt S}?}\\

\begin{enumerate}
\item[A1] 
\begin{verbatim}
print S.count("word")
\end{verbatim}
\item[A2] 
\begin{verbatim}
print S.find("word")
\end{verbatim}
\item[A3] 
\begin{verbatim}
print S.count(word)
\end{verbatim}
\item[A4] 
\begin{verbatim}
print S.parse_string("word")
\end{verbatim}
\end{enumerate}

\vspace{6mm}

\item {\em What is the way to check whether a string {\tt S} is a number?}\\

\begin{enumerate}
\item[A1] 
\begin{verbatim}
S.isdigit()
\end{verbatim}
\item[A2] 
\begin{verbatim}
S.isnumber()
\end{verbatim}
\item[A3] 
\begin{verbatim}
isdigit(S)
\end{verbatim}
\item[A4] 
\begin{verbatim}
isnumber(S)
\end{verbatim}
\end{enumerate}

\vspace{6mm}

\item {\em What is the way to replace in a text string {\tt S} a string {\tt s1} with another string {\tt s2}?}\\

\begin{enumerate}
\item[A1] 
\begin{verbatim}
find_and_replace(S, s1, s2)
\end{verbatim}
\item[A2] 
\begin{verbatim}
S.find_and_replace(s1, s2)
\end{verbatim}
\item[A3] 
\begin{verbatim}
S.replace("s1", "s2")
\end{verbatim}
\item[A4] 
\begin{verbatim}
S.replace(s1, s2)
\end{verbatim}
\end{enumerate}

\vspace{6mm}

\item {\em How can be in Python a string {\tt S} converted to uppercase?}\\

\begin{enumerate}
\item[A1] 
\begin{verbatim}
S.uppercase()
\end{verbatim}
\item[A2] 
\begin{verbatim}
S.raise()
\end{verbatim}
\item[A3] 
\begin{verbatim}
S.upper()
\end{verbatim}
\item[A4] 
\begin{verbatim}
S.capitalize()
\end{verbatim}
\end{enumerate}

\vspace{6mm}

\end{enumerate}

%%%%%%%%%%%%%%%%%%%%%%%%%%%%%%%%%%%%%%%%%%%%%%%%%%%%%%%%%%%%%%%%%%%%%

\section{Class Inheritance}

\begin{enumerate}

\item {\em When should inheritance be used in object-oriented programming?}\\

\begin{enumerate}
\item[A1] Always because it makes definitions of descendant classes shorter.
\item[A2] When some functionality is common to multiple classes.
\item[A3] When we work with geometrical objects. 
\item[A4] We should avoid it as it makes code more complicated.
\end{enumerate}

\vspace{6mm}

\item {\em What is the correct way to define a new class {\tt B} which is a descendant of 
      class {\tt A}?}\\

\begin{enumerate}
\item[A1] 
\begin{verbatim}
class B(A):
\end{verbatim}
\item[A2] 
\begin{verbatim}
class A(B):
\end{verbatim}
\item[A3] 
\begin{verbatim}
class B: public A:
\end{verbatim}
\item[A4] 
\begin{verbatim}
class B = descendant(A)
\end{verbatim}
\end{enumerate}

\vspace{6mm}

\item {\em What is the correct way to call from a descendant class {\tt B} the constructor of its parent class 
      {\tt A}?}\\

\begin{enumerate}
\item[A1] 
\begin{verbatim}
def __init__(self):
    A.__init__(self)
\end{verbatim}
\item[A2] 
\begin{verbatim}
def __init__(self):
    __init__(A)
\end{verbatim}
\item[A3] 
\begin{verbatim}
def __init__(self):
    __init__(self, A)
\end{verbatim}
\item[A4] 
\begin{verbatim}
def __init__(A):
    self.__init__(A)
\end{verbatim}
\end{enumerate}

\vspace{6mm}

\end{enumerate}

 
